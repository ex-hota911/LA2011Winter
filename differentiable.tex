\section{1回連続微分可能関数と常微分方程式}
\label{section:differentiable}

この節では定理 \ref{DifferentiableIsPspace},
つまり$(\infty, 1)$階連続微分可能な関数の常微分方程式の解は
\PSPACE 困難でありうることを示す.
\ifnum \proc = 1
ただし紙面の都合上, 詳細な証明は省き, 証明の概略を説明するに止める.
\fi

証明の流れとして, まず補題 \ref{WeakFeedback} から補題 \ref{DifferentiableFamily} を示したあと,
補題 \ref{DifferentiableFamily} から定理 \ref{DifferentiableIsPspace} を示す.
補題 \ref{DifferentiableFamily} では
補題 \ref{WeakFeedback} の関数族$(G_u)_u, (H_u)_u$ を模倣する
実関数族 $(g_u)_u, (h_u)_u$ の存在を述べている.
定理 \ref{DifferentiableIsPspace} の証明では
$(g_u)_u, (h_u)_u$ から求める $g, h$を構成する.


\subsection{多項式段一様関数族の差分方程式と\PSPACE}


任意の \PSPACE に含まれる言語 $L$ について, 
その差分方程式が $L$ を認識する
多項式段一様な関数族 $(G_u)_u$ が存在すること(補題 \ref{WeakFeedback})は
河村の論文のなかで示されているが, その証明の概略を示す.

任意の $L$ ではなく \PSPACE 完全な言語である
\textsf{QBF} を認識する $(G_u)_u$ を構成することにより,
任意の $L$ を認識する多項式段一様な関数族 $(G_u)_u$ が存在することを示す.
ここで \textsf{QBF} とは,
文字列 $u$ を $\psi = Q_1 x_1 \cdots Q_n x_n \phi(x_1, \dots, x_n)$ と解釈したとき 
$u \in \textsf{QBF} \Leftrightarrow \psi = T$ によって定義される言語である. 
ただし$Q_i$ は$\exists$ または $\forall$,  
$\phi(x_1, \dots, x_n)$ は $x_i$ 以外の変数を含まない論理式とする. 

論理式 $\psi = Q_1 x_1 \cdots Q_n x_n \phi(x_1, \dots, x_n)$ の値を
$\vee, \wedge$によってラベル付された二分木によって計算することを考える. 
量化子$Q_1 x_1$ を除き $x_1$ を $T$ と $F$ に置き換えた式をそれぞれ
$\psi_T = Q_2 x_2 \cdots Q_n x_n \phi(T, x_2, \dots, x_n)$,
$\psi_F = Q_2 x_2 \cdots Q_n x_n \phi(F, x_2, \dots, x_n)$ と置くと
$Q_1=\forall$ ならば $\psi = \psi_T \wedge \psi_F$, 
$Q_1=\exists$ ならば $\psi = \psi_T \vee \psi_F$.
つまり変数の1つ少ない2つの論理式と量化子によってもとの論理式の値も決まる.
これを再帰的に繰り返すことで $\psi$ は計算可能であり, 
それは深さ $n$ の二分木を葉から根へ値を定めていくことと同じである.
この過程は二分木の深さが段数に, 幅が列数に対応する形で
多項式段一様な関数による差分方程式で模倣可能であるため,
\textsf{QBF} を認識する多項式段一様関数の差分方程式が存在する.



\subsection{多項式段差分方程式を模倣する関数族}

 補題 \ref{DifferentiableFamily} を述べるため, 
 実関数族 $(g_u)_u \ (g_u \colon [0,1] \times [-1, 1] \to \R)$ 
 が多項式時間計算可能であることを定義する.
 ある神託機械 $M$ が実関数族 $(g_u)_u$ を計算するとは,
 実数 $t,y$ の名 $\phi, \psi$ を神託として受けとり,
 文字列 $u$, 求める精度 $n$ を入力として受けとったとき,
 $|M^{\phi, \psi}(u, 0^n) - g_u(t, y)| \le 2^{-n}$ を満たすことである.
 そして実関数族が多項式時間計算可能であるとは,
 その実関数族を計算し, $|u|$ と $n$ の多項式時間で動作する神託機械が存在することと定義する.


\begin{lemma}
 \label{DifferentiableFamily}
 任意の言語 $L \in $ \PSPACE に対して, 
 係数のみに $i$ を含む多項式 $\mu_i$ が存在して,
 任意の多項式 $\gamma$ に対して,
 多項式 $\rho$, 関数族 $(g_u)_u, (h_u)_u$ で, 
 $(g_u)_u$ は多項式時間計算可能であり,
 各二進文字列 $u$ に対して以下を満たすものが存在する.
 \begin{enumerate}
  \item $g_u\colon [0,1] \times [-1,1]\to \R, \quad h_u\colon [0,1] \to [-1,1]$;
  \item $h_u$ は $g_u$ の常微分方程式 (\ref{eq:ode}) の解; 
  \item $g_u$ は $(\infty, 1)$ 階連続微分可能;
  \item 任意の $i \in \N$, $y \in [-1,1]$ に対して
	\begin{equation*}
	 \D{i, 0} g_u(0,y) = \D{i, 0} g_u(1,y) = 0 
	\end{equation*}
  \item 任意の $i \in \N$ に対して
	\begin{align*}
	 |\D{i,1} g_u| &\leq 2^{\mu_i (|u|) - \gamma(|u|)}, &
	 |\D{i,0} g_u| &\leq 2^{\mu_i (|u|) - \gamma(|u|)}
	\end{align*} \label{enum:infty1}
  \item $h_u(1) = 2^{-\rho(|u|)}L(u)$.
 \end{enumerate}
\end{lemma}

 この補題より関数族 $(g_u)_u, (h_u)_u$ で $h_u$ は $g_u$ の常微分方程式の解であり,
 $h_u(1)$ に $L(u)$ の情報を持つものの存在が示される.

 条件 (iii) -- (v) はすべて定理 \ref{DifferentiableIsPspace} の $g$ が
 滑らかな関数とするために必要となる条件である.
 詳細については定理の証明の際に説明する.
 

 この補題の証明の基本的な流れを説明する.
 まず任意の言語 $L \in \PSPACE$ に対し, 
 補題 \ref{WeakFeedback} を用いて $L$ を認識する差分方程式 $(G_u)_u$ 
 及びその解 $(H_u)_u$ を得る.
 そして各 $G_u, H_u$ から, その差分方程式の計算過程を模倣する
 $(\infty, 1)$階連続微分可能な $g_u \colon [0,1] \times [-1, 1] \to \R$ 
 とその常微分方程式の解 $h_u \colon [0,1] \to \R$ を構成する.
 また $(G_u)_u$ の一様性から $(g_u)_u$ の多項式時間計算可能性を示す.


 ここまでの証明の流れは基本的に, リプシッツ連続条件の場合の証明とかわらない.
 違いは $g_u$ を滑らかな関数にするため, 
 以下のような滑らかな多項式時間実関数 $f \colon [0,1] \to \R$ を用いて
 $g_u$ を構成している点である.

 \begin{lemma}[補題 3.6. \cite{ko1991complexity}]
  \label{SmoothFunction}
  以下を満たす多項式時間無限回微分可能実関数 $f \colon [0,1] \to \R$ が存在する.
  \begin{enumerate}
   \item $f(0) = 0, \quad f(1) = 1$;
   \item 任意の $n \ge 1$ で $f^{(n)}(0) = f^{(n)}(1) = 0$;
   \item $f$ は $[0,1]$ で単調増加;
   \item 任意の $n \ge 1$ で $f^{(n)}$ は多項式時間実関数.
  \end{enumerate}
 \end{lemma} 

%%% 証明は削除, master に残っているものを後でマージする.


\subsection{定理 \ref{DifferentiableIsPspace} の証明}

 補題 \ref{DifferentiableFamily} から得られる
 $(g_u)_u$ と $(h_u)_u$ から滑らかな $g$ とその常微分方程式の解で
 その常微分方程式の解で \PSPACE 困難な $h$ を構成する.
 各 $h_u(1)$ には $L(u)$ の情報が含まれるため,
 すべての $h_u$ を一つの関数 $h$ に埋め込みたい,
 そこで [0,1] を無限の区間に分割し, $h$ の各文字列 $u$ に対応する区間
 $[l^-_u, c_u]$ に $h_u$ を
 縮小して埋め込む. 
 ただし次の文字列 $u'$ の計算に影響を与えないために,
 $h_u$ を定義域方向について反転したものを
 区間 $[c_u, l^+_u]$ に埋め込むことで影響を相殺する.
 つまり $h(l^-_u) = 0,\ h(c_u) = 2^{-\rho'(|u|)} L(u),\ h(l^+_u) = 0$ をみたす
 ように $h_u(t)$埋め込む.
 ただし $\rho'$ とは $\rho$ に縮小率をかけたものとする.
 同様に $g$ は $h$ が常微分方程式の解となるよう,
 各文字列 $u$ に対応する区間に $g_u$ を縮小して埋め込む.

 リプシッツ連続条件と異なる点は, $g_u$ を構成する時点で
 $(\infty, 1)$ 階連続微分可能にするために,
 $|\D{i,0} g_u|, |\D{i,0} g_u|$ の大きさを制限する点である
 [補題 \ref{DifferentiableFamily} の (\ref{enum:infty1})].
 $t=1$ 付近, つまり $|u| \to \infty$ のときに
 $\D{i,0} g_u, \D{i,1} g_u$ が発散しないためにこのような条件をつけている.
 

%%% 証明は削除, master に残っているものを後でマージする.
