\section{微分可能関数と常微分方程式}

  \subsection{離散初期値問題}
  初期値問題の離散バージョンが\PSPACE 完全であるところから始める.
  離散初期値問題とは
  定数 $d$, 多項式 $P, Q : \N \to \N$, 
  関数族 $(G_u)_u$ の組
  $\left< d, P, Q, (G_u)_u \right>$で
  以下の制限を満たすものである.
  \begin{equation}
    (G_u)_u \text{は多項式時間計算可能} \label{enum:discreteode1}
  \end{equation}
  \begin{equation}
    G_u: [P(|u|)] \times \left[ 2^{\poly Q} \right] \times [d] 
	\to \{-1, 0, 1\} \label{enum:discreteode2}
  \end{equation}
  また関数族 $(H_u)_u$ が以下を満たすとき,
  $(H_u)_u$ を離散初期値問題  $\left< d, P, Q, (G_u)_u \right>$ の解と呼ぶ.
  \begin{equation}
  H_u: [P(|u|) + 1] \times 
	\left[ 2^{\poly Q} \right] \to [d] \label{enum:discreteode3}
  \end{equation}
  \begin{align*}
   H_u(i, 0) &= H_u(0, T) = 0, \\
   H_u(i + 1, T + 1) & = H_u(i+1, T) + G_u(i, T, Hu(i, T)) \\
   & \left(i \in \left[P(|u|) \right],
   \ T \in \left[ 2^{\poly Q} \right]\right) \taghere
   \label{enum:discreteode4}
  \end{align*}
}

 離散初期値問題が言語 $L$ を認識するとはその解 $(H_u)_u$ が任意の文字列 $u$ で
 $H_u(P(|u|), 2^{Q(|u|)}) = L(u)$ を満たすこと.

 河村の論文において離散初期値問題が \PSPACE 完全であることが示されている.

 \begin{lemma}[補題 4.7. \cite{kawamura2010lipschitz}]
  \label{WeakFeedback}
  任意の言語 $L \in $ \PSPACE にたいして,
  $L$ を認識する離散初期値問題が存在する.
 \end{lemma}

 もと論文では $d=4$ であったが, 後の議論との統一のために一般化する.


\subsection{離散初期値問題を模倣する関数族}

任意の言語 $L \in $ \PSPACE, 文字列 $u$ にたいして,
上記の計算を模倣し $L(u)$ を計算する微分可能な実関数 $g_u$ を構成する.

\begin{lemma}
 \label{DifferentiableFamily}
 任意の言語 $L \in $ \PSPACE, 多項式 $\gamma$ にたいして,
 多項式 $\rho$, 多項式族 $(\mu_i)_i$ 関数族 $(g_u)_u, (h_u)_u$ で, 
 $(g_u)_u$ は多項式時間計算可能であり,
 各二進文字列 $u$ にたいして以下を満たすものが存在する.
 \begin{enumerate}
  \item $g_u:[0,1] \times [-1,1]\to \R, \quad h_u: [0,1] \to [-1,1]$;
  \item 任意の $y \in [-1,1]$ にたいして $g_u(0,y) = g_u(1,y) = 0 $; 
  \item $h_u$ は $g_u$ の常微分方程式 \ref{eq:ode} の解; 
  \item $g_u$ は $(\infty, 1)$ 階微分可能;
  \item $|\D{i,1} g_u| \leq 2^{\mu_i (|u|) - \gamma(|u|)}$, \qquad
	$|\D{i,0} g_u| \leq 2^{\mu_i (|u|) - \gamma(|u|)}$
  \item $h_u(1) = 2^{-\rho(|u|)}L(u)$.
 \end{enumerate}
\end{lemma}

 この補題の証明の前に, 葛によって示されている
 滑らかな多項式時間実関数 $f : [0,1] \to \R$ を導入する.
 
 \begin{lemma}[補題 3.6. \cite{ko1991complexity}]
  \label{SmoothFunction}
  以下を満たす多項式時間無限回微分可能実関数 $f : [0,1] \to \R$ が存在する.
  \begin{enumerate}
   \item $f(0) = 0, \quad f(1) = 1$;
   \item 任意の $n \ge 1$ で $f^{(n)}(0) = f^{(n)}(1) = 0$;
   \item $f$ は $[0,1]$ で単調増加;
   \item 任意の $n \ge 1$ で $f^{(n)}$ は多項式時間実関数.
  \end{enumerate}
 \end{lemma} 



 \begin{proof}[\rm 補題 \ref{DifferentiableFamily} の証明]
  補題 \ref{WeakFeedback} から, $L$ を認識する 
  離散初期値問題 $\left< d, P, Q,(G_u)_u,(H_u)_u \right>$
  とその解 $(H_u)_u$ を得る.
  各ステップを $P(u)$ 個に分割することで, $G_u(i, T, Y) \not = 0$ を満たす
  $i$ を各 $T$ にたいしてたかだか1つにすることができる. そのような $i$ のことを 
  $j_ul(T)$ と表現する. 任意の $i$ で $G_u(i, T, Y) = 0$ ならば $j_u(T)$ は
  任意の値を取るとする. 
  さらに以下を満たすとしても一般性を失わない.
  \begin{equation}
   H_u(i, 2^{Q(|u|)}) = \begin{cases}
			 L(u) & (i=P(|u|)) \\
			0 & (i<P(|u|))
			\end{cases}
  \end{equation}

 \begin{equation}
  G_u(i, 2\cdot 2^{Q(|u|)} - 1 - T, Y) 
   = \begin{cases}
      0 & (i=P(|u|)-1) \\
      -G_u(i,T,Y) & (i<P(|u|)-1)
     \end{cases}
 \end{equation}

 \begin{equation}
  H_u(i, 2 \cdot 2^\poly Q - T) 
  = \begin{cases}
    H_u(P(|u|), 2^{Q(|u|)}) & (i=\poly P) \\
    H_u(i, T) &  (i<P(|u|))
    \end{cases}
 \end{equation}

  補題 \ref{SmoothFunction} の $f$ にたいして, 
 自然数 $c_i$ を各自然数 $i$ にたいして 
  $|\D{i}f(x)| \le 2^{c_i}$ を満たす最小の自然数と定める.
 定数 $d' = \lceil \log (4d + 1) \rceil$, 
 $B = 2^{\poly \gamma + d'}$ とおき, 
 各 $(t, y) \in [0,1] \times [-1, 1]$ にたいして,
 自然数 $N$, $\theta \in [0,1]$, 整数 $Y$, $\eta \in [-1/4, 3/4]$ を
 $t = (T + \theta)2^{-Q(|u|)}$, $y = (Y + \eta)B^{-j_u(T)}$ を満たすように
 定める.
 
 そのとき,
 \begin{equation}
  \delta_{u, Y} (t) = \frac{2^\poly Q f'(\theta)}{B^{j_u(T)+1}} 
   G_u\left( j_u(T), T, \min \left(Y \bmod 2^{d'}\!\!\!,\ d-1 \right) \right)
 \end{equation}
 とおき $g_u, h_u$ を以下のように定義する.
 \begin{equation}
  g_u(t,y) 
  = \begin{cases}
     \delta_{u, Y}(t)& (\eta \le \frac 1 4) \\
     ( 1-f ( \frac{4\eta-1}{2})) \delta_{u, Y}(t) 
     + f ( \frac{4\eta-1}{2}) \delta_{u,Y+1}(t)
     & (\eta > \frac 1 4)
    \end{cases}
  \label{eq:gu}
 \end{equation}

 \begin{equation} 
  h_u(t) = \sum^{\poly P}_{i=0} \frac{H_u(i, T)}{B^i}  
  + \frac{f(\theta)}{B^{j_u(T)+1}} G_u(j_u(T), T, H_u(j_u(T), T)) 
  \label{eq:hu}
 \end{equation}

 上記のように定義した $g_u, h_u$ が補題\ref{DifferentiableFamily} で求める
 性質を満たすことを示す. (i), (ii) は自明. 
 $(g_u)_u$ が多項式時間計算可能であることは補題によって示される.

 $h_u$ は $g_u$ の常微分方程式の解であることを示す.
 まず $h_u$ について解析する. 
  $h_u(t) = (Y + \eta) B^{-j_u(T)}$ とおくときの, $\eta$ の範囲がどうなるか,
  つまり式(\ref{eq:gu})のどちらのケースを使うかを考える.
  式(\ref{eq:hu}) の一つ目の項において
 $i \le j_u(T)$ の合計は $B^{j_u(T)}$ の倍数なので $\eta$ に影響はない.
  $i > j_u(T)$ の合計は, 
 \begin{align*}
  \sum_{i>j_u(T)} \frac{H_u(i, T)}{B^i} 
  & \le \sum_{i>j_u(T)} \frac{d-1}{B^i} 
   = \sum_{i>j_u(T)} \frac{d-1}{B^{i-j_u(T)}}B^{-j_u(T)} \\
  & \le \sum_{i>j_u(T)} \frac{(d-1)}{(4d+1)^{i-j_u(T)}}B^{-j_u(T)} \\
  & = \frac{d-1}{4d}B^{-j_u(T)}
 \end{align*}
 二つ目の項の絶対値は
 \begin{equation}
  \left| \frac{f(\theta)}{B^{j_u(T)+1}} G_u(j_u(T), T, H_u(j_u(T), T)) \right|
  \le \frac{1}{B^{j_u(T)+1}}
  \le \frac{B^{-j_u(T)}}{4d+1}
 \end{equation}
 $(\frac{d-1}{4d} + \frac{1}{4d+1})B^{-j_u(T)} \le \frac 1 4 B^{-j_u(T)} $
  より $h_u(t) = (Y + \eta) B^{-j_u(T)}$ を満たす $\eta \in [-1/4, 1/4]$
 が存在する. このとき,
 \begin{equation}
  Y = \sum_{i=0}^{j_u(T)}H_u(i, T) \cdot B^{j_u(T) - i} .
 \end{equation}
 $B$ は $2^{d'}$ の倍数なので, 
 $\min (Y \bmod 2^{d'}\!\!\!,\ d-1) = \min (H_u(j_u), d-1) = H_u(j_u)$. 
 (\ref{eq:gu})へ$Y$ と $\eta$ を代入すると,
 \begin{align*}
   g_u(t, h_u(t)) 
  & =  \frac{2^\poly Q f'(\theta)}{B^{j_u(T)+1}}
   G_u(j_u(T), T, H_u(j_u(T), T)) \\
  & =  \D{1}h_u(t).
 \end{align*}
 よって $h_u$ は $g_u$ の常微分方程式の解.

  $g_u$ が $(\infty, 1)$ 階連続的微分可能であることを証明する.
  $\eta$ が $[-1/4, 1/4]$ と $[1/4, 3/4]$ である区間それぞれにおいて微分する.

  \begin{equation}
   \D{i}\delta_{u,Y}(t) 
    = \frac{2^{(i+1)\poly Q} \D{i+1}f(\theta)}{B^{j_u(T)+1}}
    G_u\left( j_u(T), T, \min \left(Y \bmod 2^{d'}\!\!\!,\ d-1 \right) \right)
  \end{equation}

  \begin{equation}
    \D{i,0} g_u(t, y)
     = \begin{cases}
	\D{i} \delta_{u, Y}(t) 
	\hfill (- \frac 1 4 < \eta < \frac 1 4) \\
	\left( 1-f \left(\frac{4\eta-1}{2}\right)\right) 
	\D{i} \delta_{u, Y}(t) 
	+ f \left(\frac{4\eta-1}{2}\right) \D{i} \delta_{u,Y+1}(t) \\
	\hfill (\frac 1 4 < \eta < \frac 3 4)
       \end{cases}
  \end{equation}   

  \begin{equation}
    \D{i,1} g_u(t, y)
     = \begin{cases}
	0 \hfill (- \frac 1 4 < \eta < \frac 1 4) \\
	2B^{j_u(T)}\D{1}f(\frac{4\eta - 1}2)
	(\D{i}\delta_{u,Y+1}(t)-\D{i}\delta_{u, Y}(t)) \\
	\hfill (\frac 1 4 < \eta < \frac 3 4)
       \end{cases}
  \end{equation}

  $f$ は 無限回微分可能であるため, $\delta_{u,Y}$ も無限回微分可能である.
  よって 区間 $[-1/4, 1/4]$, $[1/4, 3/4]$ において
  $\D{i,0} g_u$, $\D{i,1} g_u$ は連続. 
  $\eta = 1/4$ および  $\eta = 3/4(-1/4)$ においても連続であることは自明.
  よって $g_u$ は $(\infty, 1)$ 階連続的微分可能.



  $|\D{i,1} g_u| \leq 2^{\mu_i (|u|) - \gamma(|u|)}$ および
  $|\D{i,0} g_u| \leq 2^{\mu_i (|u|) - \gamma(|u|)}$ を示す.

  \begin{equation}
   |\D{i}\delta_{u, Y}(t)| 
    \le \left|\frac{2^{(i+1)\poly Q}\D{i+1}f(\theta)}{B^{j_u(T)+1}} \right|
    \le \frac{2^{(i+1)\poly Q + c_i}}{B^{j_u(T)+1}}\\
  \end{equation}

  $\mu_i(k) = (i+1)Q(k) + c_i + c_1 + 2$ とおくと
  $B$ の定義より

  \begin{align*}
   \left| \D{i,0} g_u \right| 
   &\le 
   |\D{i}\delta_{u, Y}(t)| 
    \le \frac{2^{(i+1)\poly Q + c_i}}{B} 
    \le 2^{\mu_i (|u|) - \gamma(|u|)}
   \taghere \\
   \left| \D{i,1} g_u \right| 
   & \le 
   2B^{j_u(T)} \left|\D{1}f\left(\frac{4\eta - 1}2\right)\right|
   \cdot \left|\D{i}\delta_{u,Y+1}(t)-\D{i}\delta_{u, Y}(t)\right| \\
   & \le
   2 B^{j_u(T)} \cdot 2^{c_1} \cdot 
   2 \cdot \frac{2^{(i+1)\poly Q + c_i}}{B^{j_u(T)+1}}\\
   & =
   \frac{2^{(i+1)\poly Q + c_i + c_1 + 2}}{B}
   \le
   2^{\mu_i (|u|) - \gamma(|u|)} . \taghere
  \end{align*}


 (vii) は 
 \begin{align*}
  h_u(1) &= \frac{H_u(\poly P, 2^\poly Q)}{B^\poly P}  \\
  &= \frac{L(u)}{2^{\poly P (\poly \gamma + d')}} \taghere
 \end{align*}
 より, $\rho(k) = P(k)(\gamma(k) + d')$ とおくと成り立つ.
 \end{proof}

\subsection{定理 \ref{DifferentiableIsPspace} の証明}

\begin{proof}
 %[定理 \ref{DifferentiableIsPspace} の証明]
 $L$ を \PSPACE 完全な言語, $\lambda(k) = 2k+2$ とおく.
 \PSPACE 完全な言語 $L$ にたいして補題 \ref{DifferentiableFamily} を用いて,
 $\rho$, $(g_u)_u$, $(h_u)_u$ を得る.
 $(g_u)_u$ は多項式時間実関数族なので, $|g_u(t, y)| \le 2^{\poly \gamma - |u|}$
 を満たすような多項式 $\gamma$ が存在する.
 各 $u$ にたいして 
 \begin{equation}
  \Lambda_u = 2^{\lambda(|u|)},\quad
   \Gamma_u = 2^{\gamma(|u|)}
 \end{equation}
 \begin{equation}
  c_u = 1-\frac{1}{2^|u|}+\frac{2\bar{u}+1}{\Lambda_u}, \quad
   l_u^\mp=c_u\mp\frac{1}{\varLambda_u}
 \end{equation}
 とおく. ただし $\bar u \in \{0, \dots, 2^{|u|} - 1\}$ は $u$ を二進数と
 して解釈した数.

 関数 $g,h$ を $t \in [0,1]$, $y \in \R$ にたいして, 下のように定義する.
 \begin{equation}
  g\left(l^\mp_u \pm \frac{t}{\Lambda_u}, \frac{y}{\Lambda_u\Gamma_u}\right) \\
   = \begin{cases}
      \pm \frac{1}{\Gamma_u} \left( g_u(t,1) 
      + \D{0,1}g_u(t, 1)(y-1) \right)&  (1<y)\\
      \pm \frac{g_u(t, y)}{\Gamma_u} & (-1 \le y \le 1) \\
      \pm \frac{1}{\Gamma_u} \left( g_u(t, -1)
      + \D{0,1} g_u(t, -1)(y+1) \right) & (y<-1)
     \end{cases}
 \end{equation}
 \begin{equation}
  h \left( l^\mp_u \pm \frac{t}{\Lambda_u} \right) 
   = \frac{h_u(t)}{\Lambda_u\Gamma_u}.
 \end{equation}
 任意の $y\in \R$ にたいして $g(1,y) = h(1) = 0$ と定義する.

 $g$と$h$が定理 \ref{DifferentiableIsPspace} で求める関数の性質を満たすことを示す.
 
 まず $g$ が多項式時間計算可能であることを示す.
 補題\footnote{準備で導入}を用いて示す.
 各有理数 $T,Y$ について $g(T, Y)$ を求めるとき,
 $T=l_u^\mp \pm t/\Lambda_u$, $Y = y/\Lambda_u\Gamma_u$ を満たすような
 $u, \pm(\mp), t, y$ は, 多項式時間で計算可能である.

 次に $g$ が $y$ に関して微分可能であり, 導関数は $y$, $t$ に関して
 連続になっていることをしめす. 
 各区間で第二引数に関して微分すると,
 \begin{equation}
  \D{0,1}g \left(l^\mp_u \pm \frac{t}{\Lambda_u}, \frac{y}{\Lambda_u\Gamma_u} \right) = \begin{cases}
		   \pm \Lambda_u \D{0,1}g_u(t,1) & (1 < y) \\
		   \pm \Lambda_u \D{0,1}g_u(t,y) & (-1 < y < 1) \\
		   \pm \Lambda_u \D{0,1}g_u(t,-1) & (y < -1).
		  \end{cases}
 \end{equation}
 よって $\D{0,1}g(t,1) = \pm \Lambda_u \D{0,1}g_u(t,1)$, 
 $\D{0,1}g(t,-1) = \pm \Lambda_u \D{0,1}g_u(t,-1)$ であり,
 $y$ に関して微分可能. $\D{0,1}g_u$は連続であるため, $y$ に関して連続は自明.
 
 $t$軸方向への連続性について.
 任意の $[0,1)$ の数はある $u$ と $t \in [0,1]$ が存在して
 $l^\mp_u \pm t/\Lambda_u$ の形で表せる.
 $t \in (0,1)$ においては $\D{0,1}g_u(t, y)$は連続であるため, $t$軸方向へ連続.
 $t = 0, 1$ のとき,
 $y \in [-1, 1]$ にたいして $g_u(0, y) = g_u(1, y) = 0$ より
 $\D{0,1}g_u(0, y) = \D{0,1}g_u(1, y) = 0$
 よって $t = 0, 1$ においても連続.
 $g(1,y) = 0$ より $\D{0,1} g(1,y) = 0$.
 また $\left| \D{0,1}g_u \right| \le 2^{\poly \lambda - |u|}$より,
 \begin{equation}
  \lim_{t \to 1-0} \left| \D{0,1}g \right|
   = \lim_{|u| \to \infty} \left| \Lambda_u\D{0,1}g_u \right| 
   \le \lim_{|u| \to \infty} \left| 2^{-|u|} \right|= 0.
 \end{equation}
 よって $\D{0,1}g$ は連続.

 $h$が$g$の常微分方程式の解であることを示す. 
 $h(0)=0, \quad \D{1}h(1) = 0 = g(1,h(1))$ は自明. 
 \begin{align*}
  \lefteqn{h'(l^\mp_u \pm t/\Lambda_u)} \\ 
  &=  \pm \frac{h'_u(t)}{\Lambda_u\Gamma_u} \\ 
  &= \pm \frac{g_u \left(t, h_u(t)\right)}{\Gamma_u} \\ 
  &= g\left(l^\mp_u \pm \frac{t}{\Lambda_u},  
	\frac{h_u(t)}{\Lambda_u\Gamma_u}\right) \\ 
  &= g\left(l^\mp_u \pm \frac{t}{\Lambda_u}, 
	h\left(l^\mp_u \pm \frac{t}{\Lambda_u}\right) \right) . \taghere
 \end{align*}

 $L$ は $h$ に還元可能であることを示す.
 \begin{equation}
  h(c_u) = \frac{h_u(1)}{\Lambda_u\Gamma_u} 
   = \frac{L(u)}{2^{\lambda(|u|)+\gamma(|u|)+\rho(|u|)}}
 \end{equation}
 つまり $R,S,T$ を以下のように定義することで, 還元可能.
 \begin{align}
  R(u,v) &= v \\
  S(u, 0^n) &= \lfloor 2^nc_u \rfloor \text{を表す文字列,} \\
  T(u) &= 0^{\lambda(|u|)+\gamma(|u|)+\rho(|u|)}
 \end{align}
 $L$ は \PSPACE 完全であるため, $h$ も \PSPACE 完全.
\end{proof}

