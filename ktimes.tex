\section{任意回微分可能関数と常微分方程式}

 任意回微分可能な関数の常微分方程式の解も, ある仮定のもと \PSPACE 完全でありうることを
 証明する.

  \subsection{フィードバックの弱い計算}
  フィードバックの弱い計算を定義する.
  フィードバックの弱い計算とは
  定数 $d$, 関数 $P : \N \to \N$, 多項式 $Q : \N \to \N$, 
  関数族 $(G_u)_u, (H_u)_u$ で, 
  5つ組 $M = \left< d, P, Q, (G_u)_u, (H_u)_u \right>$である.
 \begin{itemize}
  \item  $(G_u)_u$ は多項式時間計算可能;
  \item $P(x) = O(\log x)$ かつ多項式時間計算可能;
  \item $G_u: [P(|u|)] \times \left[ 2^{\poly Q} \right] \times [d] 
	\to \{-1, 0, 1\}$;
   \item $H_u: [P(|u|) + 1] \times 
	 \left[ 2^{\poly Q} \right] \to [d]$;
   \item 任意の $i \in \left[P(|u|) \right], 
	 \ T \in \left[ 2^{\poly Q} \right]$ にたいして 
	 \begin{itemize}
	  \item $H_u(i, 0) = H_u(0, T) = 0$
	  \item $H_u(i + 1, T + 1) = H_u(i+1, T) + G_u(i, T, Hu(i, T))$.
	 \end{itemize} 
 \end{itemize}

 フィードバックの弱い計算 $M$ が言語 $L$ を認識するとは任意の文字列 $u$ で
 $H_u(P(|u|), 2^{Q(|u|)}) = L(u)$ を満たすこと.

 \begin{hypothesis}
  \label{Hypothesis}
  任意の言語 $L \in$ \PSPACE に対して $L$ を認識するフィードバックの弱い計算が
  存在する.
 \end{hypothesis}

 つまりフィードバックの弱い計算が \PSPACE 完全であることを仮定する. 

 \subsection{離散初期値問題を模倣する関数族}

 証明の流れは1回微分可能の時と変わらない.
 任意の言語 $L \in $ \PSPACE, 文字列 $u$ にたいして,
 上記のフィードバックの弱い計算を模倣し $L(u)$ を計算する
 任意回微分可能な実関数 $g_u$ を構成する.

 \begin{lemma}
  \label{KTimesFamily}
  仮定 \ref{Hypothesis} のもと, 任意の自然数 $k \ge 2$,
  任意の言語 $L \in$ \PSPACE, 任意の多項式 $\lambda$ にたいして,
  関数 $\rho : \N \to \N$ と関数族 $g_u, h_u$ で,
  $\rho, (g_u)_u$ は多項式時間計算可能であり,
  各二進文字列 $u$ にたいして以下を満たすものが存在する.
  \begin{enumerate}
   \item $g_u:[0,1] \times [-1,1]\to \R, \quad h_u: [0,1] \to [-1,1]$;
   \item 任意の $y \in [-1,1]$ にたいして $g_u(0,y) = g_u(1,y) = 0 $;
   \item $h_u$ は $g_u$ の常微分方程式の解;
   \item $\D{0,k} g_u$ は連続;
   \item 任意の $i \in \{0, \dots, k\}$ にたいして
	 $\left|\D{0,i} g_u(t,y)\right| 
		\le \Lambda_u^{-i}2^{-|u|}$;
	 \label{enum:lem:differentiable}
   \item $h_u(1) = 2^{-\rho(|u|)}L(u)$.
  \end{enumerate}
  ただし $\Lambda_u = 2^{\lambda(|u|)}$.
 \end{lemma}
 
 \begin{proof}
  仮定 \ref{Hypothesis} より $L$ を認識する 
  $M = \left< d, P, Q,(G_u)_u,(H_u)_u \right>$
  を得る.
  さらに以下のように仮定する.
  \begin{equation}
   H_u(i, 2^{Q(|u|)}) = \begin{cases}
			L(u) & (i=P(|u|)) \\
			0 & (i<P(|u|)).
			\end{cases}
  \end{equation}

    補題 \ref{SmoothFunction} の $f$ にたいして, 
 定数 $c$ を任意の $i \in \{0, \dots, k\}$, 任意の $x \in [0,1]$ にたいして $|\D i f(x)| \le 2^c$ を満たす最小の自然数と定める.
 定数 $d' = \lceil \log (4d + 1) \rceil$, 
 $B = 2^{\poly Q + k \poly \lambda + (k - 1) + |u| + c + d' + k}$ とおき, 
 各 $(t, y) \in [0,1] \times [-1, 1]$ にたいして,
 自然数 $N$, $\theta \in [0,1]$, 整数 $Y$, $\eta \in [-1/4, 3/4]$ を
 $t = (T + \theta)2^{-Q(|u|)}$, $y = (Y + \eta)B^{-(k+1)^{j_u(T)}}$ 
  を満たすように定める.
 
 そのとき,
 \begin{equation}
  g^*_u(t, Y) = \frac{2^\poly Q \pi \sin (\theta\pi)}{2 B^{(k+1)^{(j_u(T)+1)}}}
   G_u\left( j_u(T), T, Y \bmod 2^{d'} \right)
 \end{equation}
 とおき $g_u, h_u$ を以下のように定義する.
 \begin{equation}
  g_u(t,y) 
  = \begin{cases}
     g^*_u(t, Y)& (\eta \le \frac 1 4) \\
     ( 1-f \left( 2\eta - \frac 1 2 \right)g^*_u(t, Y) 
     + f \left( 2\eta -  \frac 1 2 \right) g^*_u(t, Y+1)
     & (\eta > \frac 1 4)
    \end{cases}
  \label{ktimes:eq:gu}
 \end{equation}

 \begin{equation} 
  h_u(t) = \sum^{\poly P}_{i=0} \frac{H_u(i, T)}{B^{(k+1)^i}}
  + \frac{1-\cos(\theta\pi)}{2} 
  \cdot \frac{G_u(j_u(T), T, H_u(j_u(T), T))}{B^{(k+1)^{(j_u(T)+1)}}}
  \label{ktimes:eq:hu}
 \end{equation}

 上記のように定義した $g_u, h_u$ が補題\ref{KTimesFamily} で求める
 性質を満たすことを示す. (i) , (ii) は自明. 
 $(g_u)_u$ が多項式時間計算可能であることは補題によって示される.

 $h_u$ は $g_u$ の常微分方程式の解であることを示す.
 まず $h_u$ について解析する. (\ref{ktimes:eq:hu}) の一つ目の項において
 $i \le j_u(T)$ の合計は $B^{(k+1)^{j_u(T)}}$ の倍数. $i > j_u(T)$ の合計は, 
 \begin{align*}
  \sum_{i>j_u(T)}^{\poly P} \frac{H_u(i, T)}{B^{(k+1)^i}} 
  & \le \sum_{i>j_u(T)}^\infty \frac{d-1}{B^{(k+1)^i}}  \\
  & \le \sum_{i>j_u(T)}^\infty \frac{d-1}{B^i} 
   = \sum_{i>j_u(T)} \frac{d-1}{B^{i-j_u(T)}}B^{-j_u(T)} \\
  & \le \sum_{i>j_u(T)} \frac{(d-1)}{(4d+1)^{i-j_u(T)}}B^{-j_u(T)} \\
  & = \frac{d-1}{4d}B^{-j_u(T)}
 \end{align*}
 二つ目の項の絶対値は
 \begin{equation}
  \left| \frac{1-\cos(\theta\pi)}{2} 
   \cdot \frac{G_u(j_u(T), T, H_u(j_u(T), T))}{B^{(k+1)^{(j_u(T)+1)}}} \right|
  \le \frac{1}{B^{j_u(T)+1}}
  \le \frac{B^{-j_u(T)}}{4d+1}
 \end{equation}
 よって $h_u(t) = (Y + \eta) B^{-j_u(T)}$ を満たす $\eta \in [-1/4, 1/4]$
 が存在する. このとき,
 \begin{equation}
  Y = \sum_{i=0}^{j_u(T)}H_u(i, T) \cdot B^{j_u(T) - i} .
 \end{equation}
 $B$ は $2^{d'}$ の倍数なので, $Y \bmod 2^{d'} = H_u(j_u)$. 
 (\ref{ktimes:eq:gu})へ$Y$ と $\eta$ を代入すると,
  \begin{align*}
   g_u(t, h_u(t)) 
  & =  \frac{2^\poly Q \pi \sin(\theta\pi)}{2B^{j_u(T)+1}}
   G_u(j_u(T), T, H_u(j_u(T), T)) \\
  & =  \D{1}h_u(t). \taghere
  \end{align*}
 よって $h_u$ は $g_u$ の常微分方程式の解.


  \begin{align}
   &\D{0,i} g(t, y)  = \\
   & \begin{cases}
     0 & (\eta \le \frac 1 4) \\
     2^i  B^{i \cdot (k+1)^{j_u(T)}} \cdot \D{i} f\left(2\eta - \frac 1 2 \right)
      \cdot (g^*_u (t, Y+1)-g^*_u(t, Y))
     & (\eta > \frac 1 4)
     \end{cases} \notag
  \end{align}
  よって $\D{0,i} g$ は連続


  (v) を示す. $|g^*_u| 
  \le \left| \frac{2^\poly Q \pi \sin (\theta\pi)}{2 B^{(k+1)^{(j_u(T)+1)}}} \right|
  \le \frac{2^{\poly Q + 1}}{B^{(k+1)^{(j_u(T)+1)}}}$ より, 
  $i = 0$ において
  \begin{equation}
   |\D{0,0} g| = | g |  
    \le \frac{2^{\poly Q + 1}}{B^{(k+1)^{(j_u(T)+1)}}} 
    \le \frac{2^{\poly Q + 1}}{B^{(k+1)}} \le 2^{-|u|}
  \end{equation}
  $i \in \{1, \dots, k\}$ において,
  \begin{align*}
   | \D{0, i} g | & \le 2^i \cdot B^{i \cdot (k+1)^{j_u(T)}} \cdot 2^c \cdot
   (g^*_u (t, Y+1)-g^*_u(t, Y)) \\
   & \le 2^{c+k} \cdot B^{k(k+1)^{j_u(T)}} \cdot 2 \cdot \frac{2^{\poly Q + 1}}{B^{(k+1)^{(j_u(T)+1)}}} \\
   & \le \frac{2^{\poly Q + c + k + 2}}{B} 
   \le 2^{-i\poly \lambda - |u|}
    =  \Lambda_u^{-i}2^{-|u|} \taghere
  \end{align*}


  (vii) は 
  \begin{align*}
   h_u(1) &= \frac{H_u(\poly P, 2^\poly Q)}{B^{(k+1)^\poly P}}  \\
   &= \frac{L(u)}
   {2^{(k+1)^\poly P (\poly Q + k \poly \lambda + |u| + c + d' + k)}}   \taghere
  \end{align*}
  より, $\rho(x) = (k+1)^{P(x)} (Q(x) + k \lambda(x) + x + c + d' + k)$ とおく.
  $\poly P = O(\log |u|)$ かつ $P$ は多項式時間計算可能により, 
  $\rho$ は多項式時間計算可能.
 \end{proof}





 \subsection{定理 \ref{KTimesIsPspace} の証明}

\begin{proof}
 $L$ を \PSPACE 完全な言語, $\lambda(k) = 2k+2$ とおく.
 \PSPACE 完全な言語 $L$ にたいして補題 \ref{KTimesFamily} を用いて,
 $\rho$, $(g_u)_u$, $(h_u)_u$ を得る.
 \begin{align}
  c_u &= 1-\frac{1}{2^|u|}+\frac{2\bar{u}+1}{\Lambda_u},&
   l_u^\mp &= c_u\mp\frac{1}{\varLambda_u}
 \end{align}
 とおく. ただし $\bar u \in \{0, \dots, 2^{|u|} - 1\}$ は $u$ を二進数と
 して解釈した数.

 関数 $g,h$ を $t \in [0,1]$, $y \in \R$ にたいして, 下のように定義する.
 \begin{equation}
  g\left(l^\mp_u \pm \frac{t}{\Lambda_u}, \frac{y}{\Lambda_u}\right) \\
   = \begin{cases}
      \pm g_u(t,1) \pm \D{0,1}g_u(t, 1)(y-1) 
      &  (1<y) \\
      \pm g_u(t, y) 
      & (-1 \le y \le 1) \\
      \pm g_u(t, -1) \pm \D{0,1} g_u(t, -1)(y+1)  & (y<-1)
     \end{cases}
 \end{equation}
 \begin{equation}
  h \left( l^\mp_u \pm \frac{t}{\Lambda_u} \right) 
   = \frac{h_u(t)}{\Lambda_u}.
 \end{equation}
 任意の $y\in \R$ にたいして $g(1,y) = h(1) = 0$ と定義する.

 $g$と$h$が定理 \ref{DifferentiableIsPspace} で求める関数の性質を満たすことを示す.
 
 まず $g$ が多項式時間計算可能であることを示す.
 補題\footnote{準備で導入}を用いて示す.
 各有理数 $T,Y$ について $g(T, Y)$ を求めるとき,
 $T=l_u^\mp \pm t/\Lambda_u$, $Y = y/\Lambda_u$ を満たすような
 $u, \pm(\mp), t, y$ は, 多項式時間で計算可能である.

 次に $g$ が $y$ に関して微分可能であり, 導関数は $y$, $t$ に関して
 連続になっていることをしめす. 
 各区間で第二引数に関して微分すると $i \in {1,\dots,k}$ で
 \begin{equation}
  \D{0,i}g \left(l^\mp_u \pm \frac{t}{\Lambda_u}, \frac{y}{\Lambda_u} \right) = \begin{cases}
		   \pm \Lambda_u^i \D{0,i}g_u(t,1) & (1 < y) \\
		   \pm \Lambda_u^i \D{0,i}g_u(t,y) & (-1 < y < 1) \\
		   \pm \Lambda_u^i \D{0,i}g_u(t,-1) & (y < -1).
		  \end{cases}
 \end{equation}
 よって $\D{0,i}g(t, 1) = \pm \Lambda_u^i \D{0,i}g_u (t,1)$, 
 $\D{0,i}g(t, -1) = \pm \Lambda_u^i \D{0,i}g_u (t,-1)$ であり,
 $y$ に関して $i$ 回微分可能かつ$i$階の導関数は連続.
 
 $t$ 軸方向への $i$ 階の導関数の連続性について.
 任意の $[0,1)$ の数はある $u$ と $t \in [0,1]$ が存在して
 $l^\mp_u \pm t/\Lambda_u$ の形で表せる.
 $t \in (0,1)$ においては $\D{0,i}g_u(t, y)$は連続であるため, $t$軸方向へ連続.
 $t = 0, 1$ のとき, $y \in [-1, 1]$ にたいして $g_u(0, y) = g_u(1, y) = 0$ より
 $\D{0,i}g_u(0, y) = \D{0,i}g_u(1, y) = 0$
 よって $t = 0, 1$ においても連続.
 $g(1,y) = 0$ より $\D{0,i} g(1,y) = 0$.
 また $\left| \D{0,i}g_u \right| \le \Lambda_u^{-i} 2^{- |u|}$より,
 \begin{equation}
  \lim_{t \to 1-0} \left| \D{0,i}g \right|
   = \lim_{|u| \to \infty} \left| \Lambda_u^i \D{0,i}g_u \right| 
   \le \lim_{|u| \to \infty} \left| 2^{-|u|} \right|= 0.
 \end{equation}
 よって $\D{0,i}g$ は連続.

 $h$が$g$の常微分方程式の解であることを示す. 
 $h(0)=0, \quad \D{1}h(1) = 0 = g(1,h(1))$ は自明. 
 \begin{align*}
  \lefteqn{h'(l^\mp_u \pm t/\Lambda_u)} \\ 
  &=  \pm \frac{h'_u(t)}{\Lambda_u} \\ 
  &= \pm g_u \left(t, h_u(t)\right) \\ 
  &= g\left(l^\mp_u \pm \frac{t}{\Lambda_u},  
	\frac{h_u(t)}{\Lambda_u}\right) \\ 
  &= g\left(l^\mp_u \pm \frac{t}{\Lambda_u}, 
	h\left(l^\mp_u \pm \frac{t}{\Lambda_u}\right) \right) . \taghere
 \end{align*}

 $L$ は $h$ に還元可能であることを示す.
 \begin{equation}
  h(c_u) = \frac{h_u(1)}{\Lambda_u} 
   = \frac{L(u)}{2^{\lambda(|u|)+\rho(|u|)}}
 \end{equation}
 つまり $R,S,T$ を以下のように定義することで, 還元可能.
 \begin{align}
  R(u,v) &= v \\
  S(u, 0^n) &= \lfloor 2^nc_u \rfloor \text{を表す文字列,} \\
  T(u) &= 0^{\lambda(|u|)+\rho(|u|)}
 \end{align}
 $L$ は \PSPACE 完全であるため, $h$ も \PSPACE 完全.

\end{proof}



