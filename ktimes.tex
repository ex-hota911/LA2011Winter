\section{任意回連続微分可能関数と常微分方程式}

この節では定理 \ref{KTimesIsPspace}, 
つまり$(\infty, k)$階連続微分可能な関数の常微分方程式の解は
\DIVPlog 困難でありうることを示す.
証明の流れは節 \ref{section:differentiable}とほぼ同じである.
\ifnum \proc = 1
ただし紙面の都合上, 詳細な証明は省き, 証明の概略を説明するに止める.
\fi



\subsection{対数段差分方程式を模倣する関数族}

 \begin{lemma}
  \label{KTimesFamily}
  任意の自然数 $k \ge 2$,
  任意の言語 $L \in \DIVPlog$ に対して,
  係数のみに $i$ を含む多項式 $\mu_i$ が存在して,
  任意の多項式 $\gamma$ に対して,
  関数 $\rho \colon \N \to \N$, 関数族 $g_u, h_u$ で,
  $\rho, (g_u)_u$ は多項式時間計算可能であり,
  各二進文字列 $u$ に対して以下を満たすものが存在する.
  \begin{enumerate}
   \item $g_u\colon [0,1] \times [-1,1]\to \R, \quad h_u\colon [0,1] \to [-1,1]$;
   \item $h_u$ は $g_u$ の常微分方程式 (\ref{eq:ode}) の解;
   \item $g_u$ は $(\infty, k)$ 階連続微分可能;
   \item 任意の $i \in \N$, $y \in [-1,1]$ に対して
	 \begin{equation*}
	  \D{i, 0} g_u(0,y) = \D{i, 0} g_u(1,y) = 0 
	 \end{equation*}
   \item \label{enum:inftyk}
	 任意の $i \in \N$, $j \in \{0, \dots, k\}$ に対して
	 \begin{equation*}
	  \left|\D{i,j} g_u(t,y)\right| \le 2^{\mu_i(|u|) - \gamma(|u|)}
	 \end{equation*}
   \item $h_u(1) = 2^{-\rho(|u|)}L(u)$.
  \end{enumerate}
 \end{lemma}


補題 \ref{DifferentiableFamily} とくらべて補題 \ref{KTimesFamily} は,
\PSPACE が \DIVPlog に置き換わり, $(\infty, 1)$ 回連続微分可能が 
$(\infty, k)$回連続微分可能に一般化されている.
本質的な違いは条件 (\ref{enum:inftyk}) によって, $h$ に対するフィードバックの大きさ,
つまり $g(t,y)$ に対する $y$ の値の影響がかなり制限されてしまう点にある.
それにより, 模倣できる差分方程式が多項式段ではなく対数段に制限され,
\PSPACE 困難が \DIVPlog 困難へと置き換わる.

%%% 証明は削除, master に残っているものを後でマージする.



\subsection{定理 \ref{KTimesIsPspace} の証明}

定理 \ref{DifferentiableIsPspace} と定理 \ref{KTimesIsPspace} の関係は
補題 \ref{DifferentiableFamily} と補題 \ref{KTimesFamily} の関係と等しい.
つまり \PSPACE が \DIVPlog に置き換わり,
$(\infty, 1)$ 回連続微分可能が $(\infty, k)$回連続微分可能に一般化されている.
よって定理 \ref{DifferentiableIsPspace} の証明から
定理 \ref{KTimesIsPspace} の証明が構成できる.

%%% 証明は削除, master に残っているものを後でマージする.
