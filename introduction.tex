\section{序論}

\emph{計算可能解析学} (Computable Analysis) \cite{weihrauch00:_comput_analy} では
計算可能性理論や計算量理論の視点から解析学を扱う. 
「計算可能な実数」や「多項式時間計算可能な実関数」といった概念を定義し
(本稿では\ref{section: preliminaries}節で説明する), 
解析学に現れる様々な実数や実関数の本質的な難しさを分析する. 

連続実関数 $g \colon [0,1] \times \R \to \R$ に対して次の常微分方程式を考える. 
\begin{align}
 \label{eq:ode}
 h(0) & = 0, &
 \D h(t) & = g(t,h(t)) \quad (t \in [0,1])
\end{align}
ただし $\D h$ は $h$ の導関数.
本稿では $g$ が多項式時間計算可能であるとき, 
解 $h$ がどれほど複雑でありうるかを考える.

$g$ に多項式時間計算可能であることの他に何の制限も設けない場合, 
解 $h$ (一般に一意でない) は計算不能でありうるため,
様々な制限のもと $h$ の計算量が研究されている (表 \ref{table:related}).
この表では下に向うにつれて左列の条件が強まっている. 
Lipschitz 条件とは解 $h$ が一意であるための十分条件であり, 
これが満たされるときには, 解 $h$ は多項式領域計算可能であり, 
$\classPSPACE$ 困難でありうることがわかっている. 
つまり上界と下界が一致しているといえる (詳しくは河村 \cite{kawamura2010lipschitz}).
一方で $g$ が解析的であるとき, 解 $h$ も解析的となり, 
このとき $h$ は多項式時間計算可能である.

\begin{table*}

\renewcommand\arraystretch{1.3}
\begin{center}
 \caption{多項式時間計算可能実関数 $g$ の常微分方程式 (\ref{eq:ode}) の解 $h$ の計算量}
 \label{table:related}
 \begin{tabular}{lll}
  制限 & 上界 & 下界 \\
  \hline
   --- & --- & 計算不可能たりうる \cite{pour1979computable} \\
  $h$ が $g$ の唯一解 & 計算可能 \cite{coddington1955theory}
  & 任意の時間がかかりうる \cite{ko1983computational, miller1970recursive} \\
  $g$ が Lipschitz 条件を満たす & 多項式領域計算可能
      &	$\classPSPACE$ 困難になりうる \cite{kawamura2010lipschitz}\\
  $g$ が $(\infty, 1)$ 回連続微分可能 & 多項式領域計算可能 & \parbox[t]{14zw}{$\classPSPACE$ 困難になりうる\\{}[本稿定理\ref{DifferentiableIsPspace}]} \\
  $g$ が解析的 
  & 多項式時間計算可能 \cite{ko1988computing, kawamura2010complexity} 
  & ---
 \end{tabular}
\end{center}
\end{table*}

そこで本稿ではこの隔たりを埋めるため, 滑らか
つまり微分可能な実関数 $g$ について $h$ の計算量がどれほどになりうるかを調べた.

なお, 積分および最大化の計算量については,
実関数を無限回微分可能に制限しても一般の場合と複雑さは変わらず,
それぞれ $\mathbf{\#P}$ 困難および $\classNP$ 困難であることが示されている
(定理 3.7, 定理 5.32 \cite{ko1991complexity}).


二変数実関数 $g$ が $(i, j)$ 回連続微分可能であるとは,
第一変数について $i$ 回, 第二変数について $j$ 回微分でき, 
その導関数が連続であることとする
(\ref{section: preliminaries}節で厳密に定義).

 \begin{theorem}
  \label{DifferentiableIsPspace}
  多項式時間計算可能かつ $(\infty, 1)$ 回連続微分可能な
  実関数 $g \colon [0,1] \times [-1,1] \to \R$ であって, 
  常微分方程式\eqref{eq:ode}が
  $\classPSPACE$ 困難な解 $h \colon [0, 1] \to \R$ を持つものが存在する.
 \end{theorem}

ここで $g \colon [0,1] \times \R \to \R$ でなく
$g \colon [0,1] \times [-1, 1] \to \R$ と書いたのは, 
本稿では実関数の多項式時間計算可能性を, 
定義域が有界閉領域のときにのみ定義するからである. 
このため $h$ が区間 $[-1, 1]$ の外に値を取ることがあると
方程式(\ref{eq:ode})が意味をなさなくなるが, 
定理\ref{DifferentiableIsPspace}において $h$ が解であるというのは, 
任意の $t \in [0, 1]$ について $h (t) \in [-1, 1]$ が満たされることも含めて述べている.
Lipschitz 条件よりも強い仮定を置いているため, 
そのような $h$ は $g$ に対して, 存在すれば唯一である. 

また二変数関数 $g$ が
$i+j \le k$ を満たす任意の自然数 $i,j$ について
$(i,j)$ 回連続微分可能であることを, 
$g$ が $k$ 回連続微可能であると言うこともある.
定理\ref{DifferentiableIsPspace}で主張される $g$ は, 
$(\infty, 1)$ 回連続微分可能であるから, 
特に $1$ 回連続微分可能である. 

$2$ 回以上連続微分可能な実関数の常微分方程式の解について
$\classPSPACE$ 困難であることは $1$ 回連続微分可能と同様な方法では証明できない.
しかし $\classCH$ 困難であることが示せており別稿にて証明する予定である.
無限回微分可能な関数に対する常微分方程式の計算量は今後の課題である.

これまでの結果では関数 $g$ を多項式時間計算可能と仮定した時の
解 $h$ の計算量について解析していた.
しかしより本質的に微分方程式を「解く」困難さ, 
つまり与えられた任意の $g$ から $h$ を求める演算子の計算量について
述べる手法が河村とクックによって提案されている\cite{kawamura2010operators}.
定理\ref{DifferentiableIsPspace}も
同様に演算子の計算量として述べ直すこともできるが, 
本稿では割愛する. 
