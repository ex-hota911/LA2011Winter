\section{導入}

\subsection{計算可能解析}

\emph{計算可能解析} (Computable Analysis) \cite{weihrauch00:_comput_analy} では
計算可能性理論や計算量理論の視点から解析学を扱う. 
「計算可能な実数」や「多項式時間計算可能な実関数」といった概念を定義し, 
解析学に現れる様々な実数や実関数の本質的な難しさを分析する. 

そのために「機械が実関数$f \colon \R \to \R$を計算する」ことを定義するのであるが, 
実数は有限の文字列で表されないため,
機械がその完全な値を読み書きすることはできない.
そこで実数を近似値の列で表現する.
有理数の列 $(r_n)_n$ が $x$ を表現するとは,
$(r_n)_n$ が $x$ へ速く収束すること, 
すなわち $|r_n - x| \le 2^{-n}$ を満たすこととする.
実関数の計算とはこの数列の間の変換を行うことであり, 
厳密には神託チューリング機械によって定式化される. 
この神託機械の資源を制限することで, 
実関数が多項式時間計算可能, 或いは多項式領域計算可能であることを
定義できる.
詳しくは\ref{section: preliminaries}節で説明する. 

\subsection{問題と関連研究}

連続実関数 $g \colon [0,1] \times \R \to \R$ に対して次の常微分方程式を考える. 
\begin{align}
 \label{eq:ode}
 h(0) & = 0, &
 h'(t) & = g(t,h(t)) \quad (t \in [0,1])
\end{align}
本稿では $g$ が多項式時間計算可能であるとき, 
解 $h$ がどれほど複雑でありうるかを考える.

$g$ に他に何の制限も設けない場合, 解 $h$ (一般に一意でない) は計算不能たりうるため,
様々な制限のもと $h$ の計算量が研究されている (表 \ref{table:related}).
この表では下に向うにつれて左列の条件が強まっている. 
Lipschitz 条件とは解 $h$ が一意であるための十分条件であり, 
これが満たされるときには, 解 $h$ は多項式領域計算可能であり, 
$\classPSPACE$ 困難 (\ref{section: preliminaries}節で定義) でありうることがわかっている. 
つまり上界と下界が一致しているといえる (詳しくは河村 \cite{kawamura2010lipschitz}).
一方で $g$ が解析的であるとき, 解 $h$ も解析的となり, 
このとき $h$ は多項式時間計算可能である.

\begin{table}
\renewcommand\arraystretch{1.3}
\begin{center}
 \caption{多項式時間計算可能関数 $g$ の常微分方程式 (\ref{eq:ode}) の解 $h$ の計算量}
 \label{table:related}
 \begin{tabular}{lll}
  制限 & 上界 & 下界 \\
  \hline
   --- & --- & 計算不可能たりうる \cite{pour1979computable} \\
  $h$ が $g$ の唯一解 & 計算可能 \cite{coddington1955theory}
  & 任意の時間がかかりうる \cite{ko1983computational, miller1970recursive} \\
  $g$ が Lipschitz 条件を満たす & 多項式領域計算可能
      &	$\classPSPACE$ 困難になりうる \cite{kawamura2010lipschitz}\\
  $g$ が $(\infty, 1)$ 回連続微分可能 & 多項式領域計算可能 & \parbox[t]{14zw}{$\classPSPACE$ 困難になりうる\\{}[本稿定理\ref{DifferentiableIsPspace}]} \\
  $g$ が $(\infty, k)$ 回連続微分可能 & 多項式領域計算可能 & \parbox[t]{14zw}{$\classCH$ 困難たりうる\\{}[本稿定理\ref{KTimesIsCH}]} \\
  $g$ が解析的 
  & 多項式時間計算可能 \cite{ko1988computing, kawamura2010complexity} 
  & ---
 \end{tabular}
\end{center}
\end{table}

そこで本稿ではこの隔たりを埋めるため, 滑らかな関数, 
つまり微分可能な $g$ について $h$ の計算量がどれほどになりうるかを調べ,
以下の結果を得た.

 \begin{theorem}
  \label{DifferentiableIsPspace}
  多項式時間計算可能かつ $(\infty, 1)$ 回連続微分可能な
  実関数 $g \colon [0,1] \times [-1,1] \to \R$ であって, 
  常微分方程式(\ref{eq:ode})が
  $\classPSPACE$ 困難な解 $h \colon [0, 1] \to \R$ を持つものが存在する.
 \end{theorem}

 \begin{theorem}
  \label{KTimesIsCH}
  任意の自然数 $k \ge 2$ に対して, 
  多項式時間計算可能かつ $(\infty, k)$ 回連続微分可能な
  実関数 $g \colon [0,1] \times [-1,1] \to \R$ であって, 
  常微分方程式(\ref{eq:ode})が
  $\classCH$ 困難な解 $h \colon [0, 1] \to \R$ を持つものが存在する.
 \end{theorem}

ここで $g \colon [0,1] \times \R \to \R$ でなく
$g \colon [0,1] \times [-1, 1] \to \R$ と書いたのは, 
本稿では実関数の多項式時間計算可能性を, 
定義域が有界閉領域のときにのみ定義するからである. 
このため $h$ が区間 $[-1, 1]$ の外に値を取ることがあると
方程式(\ref{eq:ode})が意味をなさなくなるが, 
定理\ref{DifferentiableIsPspace}, \ref{KTimesIsCH}において $h$ が
解であるというのは, 
任意の $t \in [0, 1]$ について $h (t) \in [-1, 1]$ が満たされることも含めて述べている.
なお両定理とも Lipschitz 条件よりも強い仮定を置いているため, 
そのような $h$ は $g$ に対して, 存在すれば唯一である. 

 二変数関数 $g$ が $(i, j)$ 階連続微分可能であるとは,
 第一変数について $i$ 回, 第二変数について $j$ 回微分可能であり,
 その導関数が連続であることと定義する.
 この定義は一般的な多変数関数における $k$ 階連続微分可能の定義
 (任意の $k$ 階導関数が存在し, それらがすべて連続)とは異なる.

 \begin{collorary}
  多項式時間計算可能かつ $1$ 回連続微分可能な
  実関数 $g \colon [0,1] \times [-1,1] \to \R$ であって, 
  常微分方程式(\ref{eq:ode})が
  $\classPSPACE$ 困難な解 $h \colon [0, 1] \to \R$ を持つものが存在する.
 \end{collorary}

 \begin{collorary}
  任意の自然数 $k \ge 2$ に対して, 
  多項式時間計算可能かつ $k$ 回連続微分可能な
  実関数 $g \colon [0,1] \times [-1,1] \to \R$ であって, 
  常微分方程式(\ref{eq:ode})が
  $\classCH$ 困難な解 $h \colon [0, 1] \to \R$ を持つものが存在する.
 \end{collorary}

 また定理 \ref{KTimesIsCH} において
 任意の $k$ に対して $(\infty, k)$ 階微分可能な関数を考えているが,
 一つの関数が任意の $k$ に対して $k$ 階微分であることを求めているわけではない.
 つまり $g$ が無限回微分可能であると制限しているわけではない. 
 無限回微分可能な関数に対する常微分方程式の計算量は今後の課題である.


\subsection{演算子の計算量}

定理\ref{DifferentiableIsPspace}, 
\ref{KTimesIsCH}はいづれも
関数$g$を多項式時間計算可能と仮定した上で
解$h$の計算量について述べている. 
しかし微分方程式を「解く」困難さ, 
すなわち与えられた$g$から$h$を求める演算子の計算量は
如何程であろうか. 
この問に答えるには先ず
実関数を実関数へ写す演算子の計算量を
定義することを要する. 
実数の名を実数の名へ変換する機械を以て実関数の計算を定義したのと同じように, 
実関数の名を実関数の名へ変換することになる. 
ほげほげ……

\newcommand{\OpDiffIVP}{\mathit{ODE}}
\newcommand{\deltabox}{\delta _\square}
\newcommand{\classtwofont}[1]{\text{\bfseries \sffamily \upshape #1}}
\newcommand{\classFPSPACEtwo}{\classtwofont{FPSPACE}}
\newcommand{\classCHtwo}{\classtwofont{CH}}
\newcommand{\redW}{\leq _{\mathrm W}}
\newcommand{\redSW}{\leq _{\mathrm{sW}}}

\begin{theorem}
$\OpDiffIVP _1$は$(\deltabox, \deltabox)$-$\classFPSPACEtwo$-$\redW$完全. 
\end{theorem}

\begin{theorem}
$\OpDiffIVP _k$は$(\deltabox, \deltabox)$-$\classFPSPACEtwo$に属し, 
$(\deltabox, \deltabox)$-$\classCHtwo$-$\redW$困難. 
\end{theorem}

\begin{theorem}
$\OpDiffIVP _\infty$は$\classFPSPACEtwo$-$\redSW$完全. 
\end{theorem}

