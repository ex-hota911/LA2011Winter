\section{導入}

\subsection{計算可能解析}

計算可能解析 (Computable Analysis) とは計算可能性理論や計算量理論の手法を用いて解析学を行うものである. 
実数や実関数といった解析学の対象が, 機械により計算できるかを問う. 
例えば「計算可能な実数」や「多項式時間計算可能な実関数」といった概念を定義し, 
実数計算の本質的な難しさを分析する.

有限な対象においては「計算できる関数」はモデルによらず,
すべてチューリング機械で計算できるものと同値であることが知られているが,
実数計算においては, 計算できる関数が互いに異なる, いくつかのモデルが提唱されている.
その中でも本稿で扱うモデルにおいては,
「機械が実関数を計算する」ことを次のように定義する.

ある機械が実関数 $f \colon \R \to \R$ を計算するとき,
機械に入力として実数を与え, 機械は実数を返す必要がある.
しかし実数の大きさは無限であるため, 機械に与えたとしても
すべてを読みきることはできず,
また無限の長さを持つ実数をそのまま出力することはできない.

そこでまず実数を近似値の列で表現する.
有理数の列 $\{r_n\}$ が $x$ を表現するとは,
$\{r_n\}$ が $x$ へ速く収束すること, 
すなわち $|r_n - x| \le 2^{-n}$ を満たすこととする.
数列は $n \in \N$ を $r_n \in \Q$ へ移す関数と考えることもできる.
そのような関数または数列を実数の名と呼ぶ.

関数を計算する機械を考えるとき,
入力である実数は, その名を神託として機械に与える.
ある神託機械が関数 $f \colon [0, 1] \to \R$ を計算するとは,
入力となる実数 $x$ の名を神託として与えられ,
求める精度 $n$ を入力として与えられたとき,
有理数 $s_n$ で $|s_n - f(x)| \le 2^{-n}$ を満たすものを出力することとする.

この神託機械の資源を制限することで, 多項式時間({\bf P})や多項式領域(\PSPACE)に
対応する実関数のクラスを定義できる.
厳密には\ref{section: preliminaries}節において定義する.

\subsection{問題と関連研究}

連続実関数 $g \colon [0,1] \times \R \to \R$ にたいして以下のような常微分方程式を考える. 
\begin{align}
 \label{eq:ode}
 h(0) & = 0, &
 h'(t) & = g(t,h(t)) \quad (t \in [0,1])
\end{align}
この常微分方程式の難しさを, ``$g$ を単純な実関数としたとき, 
$h$ がどれほど複雑な実関数になりうるか'' という問題によって評価する.

様々な制限のもと常微分方程式の解の計算量が研究されている[表 \ref{table:related}].
$g$ に何の制限も設けない場合, その解は計算不能たりうる.
$g$ が唯一解を持つよう制限した場合, 解は計算可能であるが, 任意の時間がかかりうる.
Lipschitz 条件は常微分方程式の解の一意性を保証する重要な条件である.
Lipschitz 条件を満たすとき, 解は多項式領域計算可能であり, 多項式領域計算完全たりうる.
$g$ が解析的であるとき, 常微分方程式の解も解析的となるため, 多項式時間計算可能である.

\begin{table}
\renewcommand\arraystretch{1.5}
\begin{center}
 \caption{関連研究}
 \label{table:related}
 \begin{tabular}{llp{21zw}}
  制限 & 上界 & 下界 \\
  \hline
   --- & --- & 計算不可能たりうる \cite{todo} \\
  $h$ が $g$ の唯一解 & 計算可能 \cite{todo} & 任意の時間がかかりうる \cite{todo} \\
  Lipschitz 条件を満たす & 多項式領域 &
	  多項式領域困難になりうる \cite{kawamura2010lipschitz}\\
  $\D{0,1}g$ が連続 & 多項式領域 & 多項式領域困難たりうる[本稿] \\
  $\D{0,i}g$ が連続 & 多項式領域 & ある仮定のもと多項式領域困難たりうる[本稿] \\
  $g$ が解析的 & 多項式時間 \cite{todo} & ---
 \end{tabular}
\end{center}
\end{table}



我々は制限と下界の関係について調べ, 以下の知見を得た.

 \begin{theorem}
  \label{DifferentiableIsPspace}
  多項式時間実関数 $g \colon [0,1] \times \R \to \R$ で,
  $(\infty, 1)$ 階連続的微分可能であり,
  $g$ の常微分方程式(\ref{eq:ode})の解 $h$ が \PSPACE 完全であるものが存在する.
 \end{theorem}

 \begin{theorem}
  \label{KTimesIsPspace}
  任意の自然数 $k \ge 2$ にたいして, 
  多項式時間実関数 $g \colon [0,1] \times \R \to \R$ で, 
  $(\infty, k)$ 階連続的微分可能であり,
  $g$ の常微分方程式(\ref{eq:ode})の解 $h$ が \DIVPlog 困難であるものが存在する.
 \end{theorem}



 \DIVPlog とは本稿において定義される計算量クラスであり,
 $\DIVPlog \subseteq \PSPACE$ であるが, $\DIVPlog = \PSPACE$ は未解決であるため,
 $(\infty, k)$階連続的微分可能関数の常微分方程式の解が \PSPACE 完全になりうるかは
 未解決である.
 厳密な定義は \ref{section:divp} 節で導入する.

 二変数関数 $g$ が $(i, j)$ 階連続的微分可能であるとは,
 第一変数について $i$ 回, 第二変数について $j$ 回微分可能であり,
 その導関数が連続であることと定義する.
 これは多変数関数における $k$ 階連続的微分可能の定義とは異なる.

また定理 \ref{KTimesIsPspace} において
任意の $k$ に対して $(\infty, k)$ 階微分可能な関数を考えているが,
一つ関数が任意の $k$ にたいして $k$ 階微分ではない.
つまり $g$ が無限回微分可能であると制限しているわけではない. 
無限回微分可能な関数に対する常微分方程式の計算量は今後の課題である.





  \subsection{離散初期値問題}
  \label{section:divp}

  離散初期値問題とは
  定数 $d$, 多項式 $P, Q$,  関数族 $(G_u)_u$ の組
  $\left< d, P, Q, (G_u)_u \right>$ で
  以下の制限を満たすものである.
  \begin{equation}
    (G_u)_u \text{は多項式時間計算可能} \label{enum:discreteode1}
  \end{equation}
  \begin{equation}
    G_u\colon [P(|u|)] \times \left[ 2^{\poly Q} \right] \times [d] 
	\to \{-1, 0, 1\} \label{enum:discreteode2}
  \end{equation}
  また関数族 $(H_u)_u$ が以下を満たすとき,
  $(H_u)_u$ を離散初期値問題  $\left< d, P, Q, (G_u)_u \right>$ の解と呼ぶ.
  \begin{equation}
  H_u\colon [P(|u|) + 1] \times 
	\left[ 2^{\poly Q} + 1 \right] \to [d] \label{enum:discreteode3}
  \end{equation}
  \begin{align*}
   H_u(i, 0) &= H_u(0, T) = 0, \\
   H_u(i + 1, T + 1) & = H_u(i+1, T) + G_u(i, T, Hu(i, T)) \\
   &\left(i \in \left[P(|u|) \right],
   \ T \in \left[ 2^{\poly Q} \right]\right) \taghere
   \label{enum:discreteode4}
  \end{align*}

 離散初期値問題が言語 $L$ を認識するとはその解 $(H_u)_u$ が任意の文字列 $u$ で
 $H_u(P(|u|), 2^{Q(|u|)}) = L(u)$ を満たすことと定義する.



 河村の論文において離散初期値問題が \PSPACE 完全であることが示されている.

 \begin{lemma}[補題 4.7. \cite{kawamura2010lipschitz}]
  \label{WeakFeedback}
  任意の言語 $L \in $ \PSPACE にたいして,
  $L$ を認識する離散初期値問題が存在する.
 \end{lemma}

 もと論文では $d = 2$ であったが, 後の議論との統一のために一般化する.



  離散初期値問題へさらに制限を加えた問題を考える.
  定数 $d$, 関数 $P \colon \N \to \N$, 多項式 $Q \colon \N \to \N$, 
  関数族 $(G_u)_u$ の組 $\left< d, P, Q, (G_u)_u \right>$が
  (\ref{enum:discreteode1}), (\ref{enum:discreteode2}) に加え,
  \begin{equation}
   P(x) = O(\log x) \text{かつ $P$ は多項式時間計算可能}
  \end{equation}
  を満たすとき対数深さ離散初期値問題であると呼ぶ.
  対数深さ離散初期値問題の解 $(H_u)_u$ が満たすべき条件は離散初期値問題と同じく
  (\ref{enum:discreteode3}), (\ref{enum:discreteode4}) である.

  対数深さ離散初期値問題が認識する言語の集合を \DIVPlog と呼ぶ.
  深さが多項式である離散初期値問題が認識する言語は \PSPACE であったため,
  $\DIVPlog \subseteq \PSPACE$ であるが, $\DIVPlog = \PSPACE$ は未解決である.
  $\DIVPlog$ は $\sharp${\bf P} 困難?    
  \footnote{$d$ を $2^{R(|u|)}$ で置き換えれば, $\sharp${\bf P} 困難になる.
  離散初期値問題を定義そのように定義するか?}
