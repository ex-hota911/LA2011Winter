\section{序論}

\emph{計算可能解析学} (Computable Analysis) \cite{weihrauch00:_comput_analy} では
計算可能性理論や計算量理論の視点から解析学を扱う. 
「計算可能な実数」や「多項式時間計算可能な実関数」といった概念を定義し
(本稿では\ref{section: preliminaries}節で説明する), 
解析学に現れる様々な実数や実関数の本質的な難しさを分析する. 

連続実関数 $g \colon [0,1] \times \R \to \R$ に対して次の常微分方程式を考える. 
\begin{align}
 \label{eq:ode}
 h(0) & = 0, &
 \D h(t) & = g(t,h(t)) \quad (t \in [0,1])
\end{align}
ただし $\D h$ は $h$ の導関数.
本稿では $g$ が多項式時間計算可能であるとき, 
解 $h$ がどれほど複雑でありうるかを考える.

$g$ に多項式時間計算可能であることの他に何の制限も設けない場合, 
解 $h$ (一般に一意でない) は計算不能でありうるため,
様々な制限のもと $h$ の計算量が研究されている (表 \ref{table:related}).
この表では下に向うにつれて左列の条件が強まっており, 
$g$が(大域的) Lipschitz条件を満たせば解$h$が一意であるが, 
表の第3行にある通りこのとき唯一解$h$は
ちょうど$\classPSPACE$困難でありうることがわかっている\cite{kawamura2010lipschitz}. 
本稿の目的はより強く$g$に滑らかさの仮定を置いたときの
$h$の複雑さを調べることである. 

\begin{table}
\renewcommand\arraystretch{1.3}
\begin{center}
 \caption{$g$が多項式時間計算可能であるときの常微分方程式\eqref{eq:ode}の解$h$の計算量}
 \label{table:related}
 \begin{tabular}{lll}
  制限 & 上界 & 下界 \\
  \hline
   --- & --- & 計算不可能になりうる \cite{pour1979computable} \\
  $h$ が $g$ の唯一解 & 計算可能 \cite{coddington1955theory}
  & 任意の時間がかかりうる \cite{ko1983computational, miller1970recursive} \\
  $g$ が Lipschitz 条件を満たす & 多項式領域計算可能 \cite{ko1983computational}
      &	$\classPSPACE$ 困難になりうる \cite{kawamura2010lipschitz}\\
  $g$ が $(\infty, 1)$ 回連続微分可能 & 多項式領域計算可能 & \parbox[t]{14zw}{$\classPSPACE$ 困難になりうる\\{}[本稿定理\ref{DifferentiableIsPspace}]} \\
  $g$ が $(\infty, k)$ 回連続微分可能 & 多項式領域計算可能 & \parbox[t]{14zw}{$\classCH$ 困難になりうる\\{}[本稿定理\ref{KTimesIsCH}]} \\
  $g$ が解析的 
  & 多項式時間計算可能 \cite{muller1987uniform, ko1988computing, kawamura2010complexity} 
  & ---
 \end{tabular}
\end{center}
\end{table}

% 一方で $g$ が解析的であるとき, 解 $h$ も解析的となり, 
% このとき $h$ は多項式時間計算可能である.
% そこで本稿ではこの隔たりを埋めるため, 滑らか
% つまり微分可能な実関数 $g$ について $h$ の計算量がどれほどになりうるかを調べた.

一般に数値計算においてはしばしば, 
或る種の算法を適用できるようにするため, 
或いは解析しやすくするために, 
与えられる関数に何らかの滑らかさ (十分な回数微分可能であるなど) を仮定すると
都合のよいことがある. 
しかしこれは経験則にすぎず, 
実際に滑らかさの仮定が
解の複雑さを計算量の意味で抑える効果をもつのかについては
あまり論ぜられて来なかった. 

本稿で扱う微分方程式についていえば, 
極端なのは$g$が解析的である場合であり, 
このときにはテイラー級数として解く議論により, 
表の最下列にあるように$h$は$g$と同じく多項式時間計算可能になる. 
本稿ではLipschitz条件より強いが解析的よりは弱い滑らかさの仮定を考える
(表の第4, 5行). 
ここで$(i, j)$回連続微分可能とは, 
第一, 第二変数についてそれぞれ$i$回, $j$回微分でき, 
その導関数が連続であることである (\ref{section: preliminaries}節).

 \begin{theorem}
  \label{DifferentiableIsPspace}
  多項式時間計算可能かつ $(\infty, 1)$ 回連続微分可能な
  実関数 $g \colon [0,1] \times [-1,1] \to \R$ であって, 
  常微分方程式\eqref{eq:ode}が
  $\classPSPACE$ 困難な解 $h \colon [0, 1] \to \R$ を持つものが存在する.
 \end{theorem}

 \begin{theorem}
  \label{KTimesIsCH}
  任意の自然数 $k \ge 2$ に対して, 
  多項式時間計算可能かつ $(\infty, k)$ 回連続微分可能な
  実関数 $g \colon [0,1] \times [-1,1] \to \R$ であって, 
  常微分方程式(\ref{eq:ode})が
  $\classCH$ 困難な解 $h \colon [0, 1] \to \R$ を持つものが存在する.
 \end{theorem}

ここで $g \colon [0,1] \times \R \to \R$ でなく
$g \colon [0,1] \times [-1, 1] \to \R$ と書いたのは, 
本稿では実関数の多項式時間計算可能性を, 
定義域が有界閉領域のときにのみ定義するからである. 
このため $h$ が区間 $[-1, 1]$ の外に値を取ることがあると
方程式(\ref{eq:ode})が意味をなさなくなるが, 
定理\ref{DifferentiableIsPspace}において $h$ が解であるというのは, 
任意の $t \in [0, 1]$ について $h (t) \in [-1, 1]$ が満たされることも含めて述べている.
なお両定理とも Lipschitz 条件よりも強い仮定を置いているため, 
そのような $h$ は $g$ に対して, 存在すれば唯一である. 

$\classCH$は定数回の計数(数え上げ)によって定義される計算量クラスである
(\ref{subsection: counting hierarchy}節).

二変数関数$g$が
$i+j \le k$を満たす任意の自然数$i$, $j$について
$(i,j)$回連続微分可能であることを
$k$回連続微可能と言うこともある. 

定理\ref{KTimesIsCH}は各$k$についてそれぞれ成立つが, 
$g$が$(\infty, \infty)$回微分可能であると仮定したときの
$h$の計算量については依然として不明である. 

積分の計算量については
無限回微分可能な関数の積分も一般の関数の積分と同等に$\classNumberP$困難であるが,
解析的な関数の積分においては, 多項式時間計算可能である.
最大化でも同様に, 無限回微分可能な関数の最大化は一般の関数の最大化と同じく
$\classNP$困難であるが
\footnote{
ただし定理3.7の証明において関数$f$を
\[f(x) = 
\begin{cases}
 u_s & \text{if not } R(s,t) \\
 u_s + 2^{-(p(n)+2n+1)\cdot n} \cdot h_1(2^{p(n)+2n+1} (x - y_{s,t})) & \text{if } R(s,t)
\end{cases}\]
に修正する必要がある.
},
解析的な関数の最大化は多項式時間計算可能である\cite{ko1991complexity}.
つまり無限回微分可能と解析的という制限の間に大きな隔たりをもつ.




