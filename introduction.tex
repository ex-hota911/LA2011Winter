\section{導入}

\subsection{計算可能解析}

計算可能解析 (Computable Analysis) では計算可能性理論や計算量理論の視点から解析学を扱う. 
「計算可能な実数」や「多項式時間計算可能な実関数」といった概念を定義し, 
解析学に現れる様々な実数や実関数の本質的な難しさを分析する. 

有限な対象においては「計算できる関数」はモデルによらず,
すべてチューリング機械で計算できるものと同値であることが知られているが,
実数計算においては, 計算できる関数が互いに異なる, いくつかのモデルが提唱されている.
本稿では,
「機械が実関数を計算する」ことを次のように定義する.

実関数 $f \colon \R \to \R$ を計算する機械を考えるとき,
問題となるのが実数は有限の文字列で表されないため,
機械が完全な値を読み, 書くことはできない点である.

そこで実数を近似値の列で表現する.
有理数の列 $(r_n)_n$ が $x$ を表現するとは,
$(r_n)_n$ が $x$ へ速く収束すること, 
すなわち $|r_n - x| \le 2^{-n}$ を満たすこととする.
数列は $n \in \N$ を $r_n \in \Q$ へ移す関数と考えることもでき,
そのような関数または数列を実数の名と呼ぶ.

関数を計算する機械にたいして,
入力である実数は, その名つまり近似値を返す関数を神託として機械に与える.
そしてある神託機械が関数 $f \colon [0, 1] \to \R$ を計算するとは,
入力となる実数 $x$ の名を神託として与えられ,
求める精度 $n$ を入力として与えられたとき,
有理数 $s_n$ で $|s_n - f(x)| \le 2^{-n}$ を満たすものを出力することとする.

この神託機械の資源を制限することで, 多項式時間({\bf P})や多項式領域(\PSPACE)に
対応する実関数のクラスを定義できる.
厳密には\ref{section: preliminaries}節において定義する.

\subsection{問題と関連研究}

連続実関数 $g \colon [0,1] \times \R \to \R$ にたいして次の常微分方程式を考える. 
\begin{align}
 \label{eq:ode}
 h(0) & = 0, &
 h'(t) & = g(t,h(t)) \quad (t \in [0,1])
\end{align}
本稿では「$g$ を単純な実関数としたとき, 
$h$ がどれほど複雑な実関数になりうるか」を考える.

$g$ に何の制限も設けない場合, その解は計算不能たりうるため,
様々な制限のもと常微分方程式の解の計算量が研究されている[表 \ref{table:related}].
Lipschitz 条件とは常微分方程式の解の一意性を保証する重要な条件である.
Lipschitz 条件を満たすとき, 解は多項式領域計算可能であり, 多項式領域計算困難たりうる.
一方で $g$ が解析的であるとき, 常微分方程式の解も解析的となるため, 多項式時間計算可能である.

\begin{table}
\renewcommand\arraystretch{1.5}
\begin{center}
 \caption{関連研究}
 \label{table:related}
 \begin{tabular}{llp{21zw}}
  制限 & 上界 & 下界 \\
  \hline
   --- & --- & 計算不可能たりうる \cite{pour1979computable} \\
  $h$ が $g$ の唯一解 & 計算可能 \cite{coddington1955theory}
  & 任意の時間がかかりうる \cite{ko1983computational} \cite{miller1970recursive} \\
  Lipschitz 条件を満たす & \PSPACE
      &	\PSPACE 困難になりうる \cite{kawamura2010lipschitz}\\
  $\D{0,1}g$ が連続 & \PSPACE & \PSPACE 困難たりうる[本稿] \\
  $\D{0,i}g$ が連続 & \PSPACE & \PSPACE 困難たりうる[本稿] \\
  $g$ が解析的 
  & 多項式時間 \cite{ko1988computing} \cite{kawamura2010complexity} 
  & ---
 \end{tabular}
\end{center}
\end{table}

そこで我々はこの隔たりを埋めることを目的として, 滑らかな関数, 
つまり微分可能な $g$ について $h$ の計算量がどれほどになりうるかを調べ,
以下の結果を得た.

 \begin{theorem}
  \label{DifferentiableIsPspace}
  多項式時間実関数 $g \colon [0,1] \times \R \to \R$ で,
  $(\infty, 1)$ 回連続的微分可能であり,
  $g$ の常微分方程式(\ref{eq:ode})の解 $h$ が \PSPACE 完全であるものが存在する.
 \end{theorem}

 \begin{theorem}
  \label{KTimesIsPspace}
  任意の自然数 $k \ge 2$ にたいして, 
  多項式時間実関数 $g \colon [0,1] \times \R \to \R$ で, 
  $(\infty, k)$ 回連続的微分可能であり,
  $g$ の常微分方程式(\ref{eq:ode})の解 $h$ が \DIVPlog 困難であるものが存在する.
 \end{theorem}

 $\DIVPlog$ とは本稿において定義される計算量クラスであり,
 $\DIVPlog \subseteq \PSPACE$ であるが, $\DIVPlog = \PSPACE$ は未解決であるため,
 $(\infty, k)$階連続的微分可能関数の常微分方程式の解が \PSPACE 完全になりうるかは
 未解決である.
 厳密な定義は次節で導入する.

 二変数関数 $g$ が $(i, j)$ 階連続的微分可能であるとは,
 第一変数について $i$ 回, 第二変数について $j$ 回微分可能であり,
 その導関数が連続であることと定義する.
 この定義は一般的な多変数関数における $k$ 階連続的微分可能の定義
 (任意の $k$ 階導関数が存在し, それらがすべて連続)とは異なる.
 $i \ge j$ とすると $(i,j)$ 回連続的微分可能であれば $j$ 回連続的微分可能であるため,
 定理 \ref{DifferentiableIsPspace}, \ref{KTimesIsPspace} は
 それぞれ 1 回連続的微分可能, $k$ 階連続的微分可能と置き換えても成り立つ.

 また定理 \ref{KTimesIsPspace} において
 任意の $k$ に対して $(\infty, k)$ 階微分可能な関数を考えているが,
 一つ関数が任意の $k$ にたいして $k$ 階微分ではない.
 つまり $g$ が無限回微分可能であると制限しているわけではない. 
 無限回微分可能な関数に対する常微分方程式の計算量は今後の課題である.





\subsection{差分方程式}
\label{section:divp}

定理 \ref{DifferentiableIsPspace} と定理 \ref{KTimesIsPspace} の証明の中では,
まず滑らかな実関数の常微分方程式で模倣可能であるような離散版常微分方程式を考え, 
その「離散版常微分方程式」がある計算量クラスについて困難であることを示している.
この節ではその離散版常微分方程式である, 
差分方程式と差分方程式に対応する計算量クラスについて定義する.


$[n] = \{0, \dots , n-1\}$ と表記する.
関数 $G \colon [P] \times [Q] \times [R] \to \{-1, 0, 1\}$ にたいして,
関数 $H \colon [P + 1] \times [Q+1] \to [R]$ が
任意の $i \in [P],\ T \in [Q]$ について以下を満たすとき,
$H$ を $G$ の差分方程式の解と呼ぶ.
\begin{gather}
   H(i, 0) = H(0, T) = 0 
\\
   H(i + 1, T + 1) = H(i+1, T) + G(i, T, H(i, T))  \label{eq:divp}
\end{gather}
$P, Q, R$ のことをそれぞれ行数, 列数, 欄の大きさと呼ぶ.
$G$ と $H$ が常微分方程式の $g$ と $h$ に対応し,
$H(i, 0) = 0$ と言う条件が $h(0) = 0$,
式 (\ref{eq:divp}) と同値である $H(i + 1, T + 1) - H(i+1, T) = G(i, T, H(i, T))$
と言う条件が $h'(t) = g(t, h(t))$ と対応する.


 関数族 $(H_u)_u$ が 関数族 $(G_u)_u$ の差分方程式族の解であるとは,
 各 $u$ にたいして $G_u$ の段数が $|u|$ の多項式で抑えられ, 
 列数及び欄の大きさが $|u|$ の指数$(2^{\mathrm{poly} (|u|)})$ で抑えられており,
 $H_u$ が $G_u$ の差分方程式の解となっていることと定義する.

 言語 $L$ にたいして, 
 \begin{equation}
	L(u) = \begin{cases} 
	       1 & u \in L \\
	       0 & u \not \in L 
	       \end{cases}
 \end{equation}
 と表記する.
 言語 $L$ が差分方程式族 $(G_u)_u$ の解 $(H_u)_u$ によって認識されるとは,
 各 $u$ にたいして $G_u$ の段数と列数を $P_u, Q_u$ としたとき,
 $H_u(P_u, Q_u) = L(u)$ をみたすこととする.



 河村の論文において多項式時間差分方程式族によって認識される言語のクラスは
 \PSPACE であることが示されている.

 \begin{lemma}[補題 4.7. \cite{kawamura2010lipschitz}]
  \label{WeakFeedback}
  任意の言語 $L$ について以下は同値.
  \begin{itemize}
   \item  $L \in  \PSPACE$
   \item その差分方程式の解が $L$ を認識する,
	 多項式時間関数族 $(G_u)_u$ が存在する.
  \end{itemize}
 \end{lemma}

 ここで関数族 $(G_u)_u$ が多項式時間計算可能であるとは,
 $u$ も入力であるとみなした関数 $G$ が多項式時間計算可能なことである.



 さらに段数が対数によって制限される差分方程式を考える.
 段数が $|u|$ の対数で抑えられる $(G_u)_u$ の差分方程式を
 対数段差分方程式とよび, 多項式時間関数族 $(G_u)_u$ の対数段差分方程式
 によって認識される言語のクラスを \DIVPlog と名付ける.
 
 段数が多項式である離散初期値問題が認識する言語は \PSPACE であったため,
 $\DIVPlog \subseteq \PSPACE$ であるが, $\DIVPlog = \PSPACE$ か否かは未解決である.
 段数が2段の差分方程式を考えると,
 $H_u$ の最後の欄の値は
 $|u|$ の多項式サイズの 各 $t \in \{0,1\}^*$ について
 $G_u(0, t, 0) \in \{-1, 0, 1\}$ の和になっている.
 つまり $\sharp${\bf P} と同程度の能力を持つと言える.
 よって \DIVPlog は$\sharp${\bf P} よりも強いクラスであると考えられる.
