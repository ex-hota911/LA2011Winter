\section{導入}

\subsection{計算可能解析}

計算可能解析 (Computable Analysis) とは計算可能性理論や計算量理論の手法を用いて解析学を行うものである. 

計算可能解析における, 実数や実関数のモデル化を形式張らずに説明する.
まず実数 $x$ は $x$ へ収束する有理数の列$\{r_n\}$で表現される. 
ここで収束することの条件として $|r_n - x| \le 2^{-n}$ を用いる.
有理数 $r_n$ は符号化可能であるため, 実数は入力として精度を受け取り,
近似値を返す関数としてモデル化できる.

実数を関数としてモデル化するため, 実関数 $f:\R \to \R$ は関数を関数へ写す
高階関数となる. よって計算可能なモデル化は単純にはできない.
そこで関数 $f:\R \to \R$ が計算可能であるとは,
入力として実数 $x$ と求める精度 $n$ を与えられ,
有理数 $s_n$ で $|s_n - f(x)| \le 2^{-n}$ を満たすものを出力する機械 $M$ が存在することとする.
$x$ は有限サイズの表現を持たないため, $M$ が直接 $x$ を読むことはできない.
そこで $M$ を神託機械とし, 近似値を返す関数を神託として与える.
この神託機械の資源を制限することで, 多項式時間({\bf P})や多項式領域(\PSPACE)に
対応する実関数のクラスを定義できる.
厳密な定義については 2 章において定義する.

\subsection{問題と関連研究}


連続実関数 $g : [0,1] \times \R \to \R$ にたいして以下のような常微分方程式を考える. 
\begin{align}
 \label{eq:ode}
 h(0) & = 0, &
 h'(t) & = g(t,h(t)) \quad (t \in [0,1])
\end{align}
常微分方程式の難しさを, ``$g$ を単純な実関数としたとき, 
$h$ がどれほど複雑な実関数になりうるか'' という問題によって評価する.

様々な制限のもと常微分方程式の解の計算量が研究されている[表 \ref{table:related}].
$g$ に何の制限も設けない場合, その解は計算不能たりうる.
$g$ が唯一解を持つよう制限した場合, 解は計算可能であるが, 任意の時間がかかりうる.
Lipschitz 条件は常微分方程式の解の一意性を保証する重要な条件である.
Lipschitz 条件を満たすとき, 解は多項式領域計算可能であり, 多項式領域計算完全たりうる.
$g$ が解析的であるとき, 常微分方程式の解も解析的となるため, 多項式時間計算可能である.

{\renewcommand\arraystretch{1.5}
\begin{table}
 \caption{関連研究}
 \label{table:related}
 \begin{tabular}{ccp{15zw}}
  制限 & 上界 & \hfil 下界 \hfil \\
  \hline
   --- & --- & 計算不可能たりうる \cite{todo} \\
  $h$ が $g$ の唯一解 & 計算可能 \cite{todo} & 任意の時間がかかりうる \cite{todo} \\
  Lipschitz 条件を満たす & 多項式領域 &
	  多項式領域困難になりうる \cite{kawamura2010lipschitz}\\
  $\D{0,1}g$ が連続 & 多項式領域 & 多項式領域困難たりうる[本紙] \\
  $\D{0,i}g$ が連続 & 多項式領域 & ある仮定のもと多項式領域困難たりうる[本紙] \\
  $g$ が解析的 & 多項式時間 \cite{todo} & ---
 \end{tabular}
\end{table}

我々は制限と下界の関係について調べ, 以下の知見を得た.

 \begin{theorem}
  \label{DifferentiableIsPspace}
  多項式時間実関数 $g(t, y)$ で, 第二変数について連続的微分可能であり,
  $g$ の常微分方程式(\ref{eq:ode})の解 $h$ が \PSPACE 完全であるものが存在する.
 \end{theorem}

 \begin{theorem}
  \label{KTimesIsPspace}
  仮定 \ref{Hypothesis} のもと, 任意の自然数 $k \ge 2$ にたいして, 
  多項式時間実関数 $g(t, y)$ で, 第二変数について $k$ 階連続的微分可能であり,
  $g$ の常微分方程式(\ref{eq:ode})の解 $h$ が \PSPACE 完全であるものが存在する.
 \end{theorem}

二変数関数 $g$ が $i$ 番目の変数に関して $k$ 階連続的微分可能であるとは,
$\D^{0,k}g$ が存在し連続であることとする.
これは任意の $i_1, \dots, i_n(\sum i_j = k)$ にたいして,
$\D^{i_1, \dots, i_n}$ が連続である関数類 $C^k$ とは異なり
(包含される), 
$g$ を注目する変数のみの関数とみなしたときの連続的微分可能とも異なる.

また定理 \ref{KTimesIsPspace} において
任意の $k$ に対して $k$ 階微分可能な関数を考えているが,
一つ関数が任意の $k$ にたいして $k$ 階微分ではない.
つまり $g$ が無限回微分可能であると制限しているわけではないことに留意したい. 
無限回微分可能な関数に対する常微分方程式の計算量は今後の課題である.
