\appendix


\section{Proofs of Lemmas~\ref{KTimesFamily} and \ref{DifferentiableFamily}}
\label{appendix:proof-family}

In Lemmas~\ref{KTimesFamily} and \ref{DifferentiableFamily}, 
we have the new conditions (\ref{enum:differentiability})--(\ref{enum:smooth})
about the smoothness and the derivatives of $g_u$ 
that were not present in \cite[Lemma 4.1]{kawamura2010lipschitz}.
To satisfy these conditions,
we construct $g_u$ using the smooth function $f$ in following lemma.

\begin{lemma}[{\cite[Lemma 3.6]{ko1991complexity}}]
 \label{SmoothFunction}
 There exist a polynomial-time function $f \colon [0, 1] \to \R$ of class $\classC^\infty$ and a polynomial $s$ such that
  \begin{enumerate}
   \item $f(0) = 0$ and $f(1) = 1$;
   \item $\D ^n f (0) = \D ^n f (1) = 0$ for all $n \ge 1$;
   \item $f$ is strictly increasing;
   \item $\D ^n f$ is polynomial-time computable for all $n \ge 1$;
   \item \label{enum:polynomial-size}
	 $|\D^n f| \le s(n)$ for all $n \ge 1$. 
  \end{enumerate}
 \end{lemma}

Although the existence of the polynomial~$s$ satisfying the condition~(\ref{enum:polynomial-size}) is not stated in \cite[Lemma 3.6]{ko1991complexity},
it can be shown easily.

We prove Lemma~\ref{KTimesFamily} 
and omit the analogous and easier proof of Lemma~\ref{DifferentiableFamily}.

\begin{proof}[Lemma~\ref{KTimesFamily}]
Let $L$ and $(G_u)_u$ be as in Lemma~\ref{DIVPlogIsCHhard},
and let a function family $(H_u)_u$ be the solution of the difference equation given by $(G_u)_u$.

By a similar argument to the beginning of the proof of \cite[Lemma 4.1]{kawamura2010lipschitz},
we may assume that there exist polynomial-time functions $p$, $j_u$
and polynomials $q$, $r$ satisfying the following properties:
\begin{gather}
 G_u \colon [p(|u|)] \times [2^{q(|u|)}] \times [2^{r(|u|)}] \to \{-1, 0, 1\},
 \\
 H_u(i, 2^{q(|u|)}) = \begin{cases}
		       L(u) & \text{if} \ i=p(|u|), \\
		       0 & \text{if} \ i<p(|u|), 
		      \end{cases}
 \\
 G_u(i, T, Y) \neq 0 \to i = j_u(T).
\end{gather}
Since $G_u$ has logarithmic height,
there exists a polynomial $\sigma$ such that $(k+1)^{p(x)} \le \sigma(x)$.


We construct the families of real functions $(g_u)_u$ and $(h_u)_u$ simulating $G _u$ and $H _u$ 
in the sense that $h_u(T/2^{q(|u|)}) = \sum^{p(|u|)}_{i = 0}H_u(i, T)/B^{d_u(i)}$, 
where the constant $B$ and the function $d_u \colon [p(|u|)+1] \to \N$ are 
defined by
  \begin{align}
   B &= 2^{\gamma(|u|) + r(|u|) + s(k) + k + 3}, 
   &
   d_u(i) &= 
   \begin{cases}
    \sigma(|u|) & \text{if} \ i=p(|u|), 
    \\
    (k+1)^i & \text{if} \ i<p(|u|).
   \end{cases}
  \end{align}
For each $(t, y) \in [0,1] \times [-1, 1]$,
there exist unique $N \in \N$, $\theta \in [0,1)$, $Y \in \Z$ and $\eta \in [-1/4, 3/4)$
such that $t = (T + \theta)2^{-q(|u|)}$ and $y = (Y + \eta)B^{-d_u(j_u(T))}$.
Using $f$ and a polynomial $s$ of Lemma~\ref{SmoothFunction},
we define 
$\delta_{u,Y} \colon [0,1] \to \R$,
$
g _u \colon [0, 1] \times [-1, 1] \to \R
$ and $
h _u \colon [0, 1] \to [-1, 1]
$ by
  \begin{align}
    \label{eq:delta}
   \delta_{u, Y} (t) &= \frac{2^{q(|u|)} \D f(\theta)}{B^{d_u(j_u(T)+1)}} 
   G_u \bigl( j_u(T), T, Y \bmod 2^{r(|u|)} \bigr),
   \\
  \label{eq:gu}
  g_u(t,y) 
  &= \begin{cases}
     \delta_{u, Y}(t)
     & \text{if} \ \eta \le \frac 1 4, 
     \\
     ( 1-f ( \frac{4\eta-1}{2})) \delta_{u, Y}(t)
     + f ( \frac{4\eta-1}{2}) \delta_{u,Y+1}(t)
     & \text{if} \ \eta > \frac 1 4,
    \end{cases}
   \\
  h_u(t) 
   &= \sum^{p(|u|)}_{i=0} \frac{H_u(i, T)}{B^{d_u(i)}}  
  + \frac{f(\theta)}{B^{d_u(j_u(T)+1)}} G_u \bigl( j_u(T), T, H_u(j_u(T), T) \bigr) .
  \label{eq:hu}
  \end{align}

We will verify that $(g_u)_u$ and $(h_u)_u$ defined above satisfy all the conditions stated in Lemma~\ref{KTimesFamily}.
Polynomial-time computability of $(g_u)_u$ can be verified using Lemma~\ref{lem:type1representation}.
The condition~(\ref{enum:kf:start}) is immediate from \eqref{eq:gu} and \eqref{eq:hu}.
The condition~(\ref{enum:equation}) is verified by a similar argument to
\cite[Lemma 4.1]{kawamura2010lipschitz}.

It is easy to verify that $g_u$ is of class $\classC^{(\infty, \infty)}$ 
since we construct $\delta_u$ from $G_u$ and $g_u$ from $\delta_u$ by connecting them with $f$ of class $\classC^{(\infty, \infty)}$.
This means that the condition~(\ref{enum:differentiability}) holds.
The derivative $\D^i \delta$ is given by for each $i \in \N$,
\begin{equation}
    \D^i \delta_{u,Y}(t) 
    = \frac{2^{(i+1)q(|u|)} \D^{i+1}f(\theta)}{B^{d_u(j_u(T)+1)}}
    G_u\bigl( j_u(T),\ T,\ Y \bmod 2^{r(|u|)} \bigr).
\end{equation}
The derivative $\D^{(i,j)} g$ is given by for each $i \in \N$ and $j=0$,
\begin{equation}
     \D^{(i, 0)} g_u(t, y)
     = \begin{cases}
 	\D^i \delta_{u, Y}(t) 
	& \text{if} \ \eta \le \frac 1 4, 
	\\
	\left( 1-f \left(\frac{4\eta-1}{2}\right)\right) 
	\D^i \delta_{u, Y}(t)
	+ f \left(\frac{4\eta-1}{2}\right) \D^i \delta_{u,Y+1}(t) 
	& \text{if} \ \frac 1 4 < \eta,
       \end{cases} \label{eq:d(i,0)g_u}
\end{equation}
and for each $i \in \N$ and $j \in \{1, \dots, k\}$,
  \begin{equation} \label{eq:d(i,j)g_u}
    \D^{(i, j)} g_u(t, y)
     = \begin{cases}
	0 & \text{if} \ {-\frac 1 4} < \eta < \frac 1 4, \\
	(2B^{d_u(j_u(T))})^j \D^j f(\frac{4\eta - 1}2)
	(\D^i \delta_{u,Y+1}(t)-\D^i \delta_{u, Y}(t)) 
	& \text{if} \ \frac 1 4 < \eta < \frac 3 4.
       \end{cases}
  \end{equation}
Substituting $t = 0, 1$ ($\theta = 0$) into \eqref{eq:d(i,0)g_u},
we get $\D^{(i, 0)} g_u(0,y) = \D^{(i, 0)} g_u(1,y) = 0$, 
so the condition~(\ref{enum:boundary}) holds.

We show that the condition~(\ref{enum:smooth}) holds with $\mu(x, y) = (x+1)q(y) + s(x+1)$.
Note that $\mu$ is a polynomial and independent of $k$ and $\gamma$.
Since $|\D^i \delta_{u,Y}(t)| \le 2^{(i+1)q(|u|) + s(i+1)}B^{-d_u(j_u(|u|)+1)}$ by \eqref{eq:delta}, for all $i \in \N$ and $j \in \{0, \dots, k\}$, we have
 \begin{equation}
  |\D^{(i,j)} g_u| 
   \le 
   2^k B^{k \cdot j_u(T)} 2^{s(k)} \cdot 2 \cdot 
   \frac{2^{(i+1)q(|u|) + s(i+1)}}{B^{d_u(j_u(|u|)+1)}} 
   \le
   \frac{2^{\mu(i, |u|) + s(k) + k + 1}}{B}
   \le
   2^{\mu(i, |u|) - \gamma(|u|)}
 \end{equation}
by \eqref{eq:d(i,0)g_u}, \eqref{eq:d(i,j)g_u} and our choice of $B$.

We have (\ref{enum:kf:end}) with
  $\rho(x) = \sigma(x) \cdot (\gamma(x)+r(x)+s(k)+k+3)$, because
  \begin{equation}
   h_u(1) = \frac{H_u(p(|u|), 2^{q(|u|)})}{B^{d_u(p(|u|))}} 
          = \frac{L(u)}{2^{\sigma(|u|) \cdot (\gamma(|u|)+r(|u|)+s(k)+k+3)}}
	  = 2^{-\rho(|u|)} L(u).
  \end{equation}
\qed
\end{proof}



 To prove Lemma~\ref{DifferentiableFamily}, 
 let $L$ and $(G_u)_u$ be as Lemma~\ref{DIVPpolyIsPSPACEhard},
 and let $(H_u)_u$ be the solution of the difference equation given by $(G_u)_u$.
 Define $(g_u)_u$ and $(h_u)_u$ as \eqref{eq:gu} and \eqref{eq:hu}
 with $d_u(i) = i$.
 It is shown in the same way as above that they meet all the conditions
 stated in Lemma~\ref{DifferentiableFamily}.
