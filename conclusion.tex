\section{考察}


\subsection{議論}
一回微分可能な関数の常微分方程式の解は \PSPACE 完全足りうることを本稿では示したが,
二回微分可能以上の制限を設けた場合に関しては仮定 \ref{Hypothesis} を前提としている.
しかし一回微分可能と二回微分可能の間にそんな本質的なギャップがあるとは思えず,
二回微分可能以上に関しても \PSPACE 完全足りうることを証明できるのではないかと考えている.
証明される可能性としてひとつは仮定 \ref{Hypothesis} が示されることであるが,
重要なのは仮定が否定されたとしても定理 \ref{KTimesIsPspace} が否定されるわけではない
ということである.

問題は仮定 \ref{Hypothesis} は妥当か.
フィードバックの深さは対数領域しか使えないが, $G_u$ は入力として時刻 $T \in [2^\poly Q]$
を持つため, 対数領域しか使えないわけではない.


\subsection{課題}


任意階微分可能な関数の常微分方程式の解が \PSPACE 完全たりうることを
証明することが第一の課題である.
仮定 \ref{Hypothesis} の証明が可能ならば, 定理 \ref{KTimesIsPspace} から,
``仮定 \ref{Hypothesis} のもと,'' という文言は取り払われ, 証明される.
しかし, フィードバックの弱い計算が任意階微分可能な関数の常微分方程式で
模倣できる計算の最大限であるという保証はない.
つまりまたはまったく別の \PSPACE 完全な計算を,
任意回微分可能な関数の常微分方程式で模倣できる可能性も残っている.

依然として解が多項式時間実関数となる, 解析的であるという条件との間にはギャップが存在し,
例えば $g$ が無限回連続微分可能でかつであるとき, 解はどうなるのか等の疑問が生まれる.

また $g$ の第一引数 $t$ に関して本稿では連続であることのみ要求したが,
微分可能になると解はどうなるか, 更に制限するとどうなるかは不明である.



