\section{考察}


\subsection{議論}
$(\infty, 1)$回連続微分可能な関数の常微分方程式の解は \PSPACE 困難たりうることを本稿では示したが,
しかし一回連続微分可能と二回連続微分可能の間に本質的なギャップがあるとは思えず,
$(\infty, k)$回連続微分可能以上に関しても \PSPACE 困難たりうることを証明できるのではないかと考えている.
証明される可能性としてひとつは $\PSPACE = \DIVPlog$ が示されることであるが,
重要なのは $\PSPACE \not = \DIVPlog$ ならば $(\infty, k)$ 回連続微分可能
な関数の常微分方程式の解が \PSPACE 困難になりえないというわけではないことである.


\subsection{課題}

任意階微分可能な関数の常微分方程式の解が \PSPACE 困難たりうることを
証明することが第一の課題である.
しかし, 対数深さ離散初期値問題が任意階微分可能な関数の常微分方程式で
模倣できる計算の最大限であるという保証はない.
つまりまたはまったく別の \PSPACE 困難な計算を,
任意回微分可能な関数の常微分方程式で模倣できる可能性も残っている.

依然として解が多項式時間実関数となる, 解析的であるという条件との間にはギャップが存在し,
例えば $g$ が無限回連続微分可能でかつであるとき, 解はどうなるのか等の疑問が生まれる.

また $g$ の第一引数 $t$ に関して本稿では連続であることのみ要求したが,
微分可能になると解はどうなるか, 更に制限するとどうなるかは不明である.



