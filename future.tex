\section{今後の課題}
一つ目の課題は$(\infty, k)$回連続微分可能という制限の元での
タイトな計算量を見つけることである.
$\classPSPACE = \classCH$でない限り, 未だ上界と下界の間にはギャップが存在する.

一つの方針は$\quantC_{\log} B_{be}$にカープ帰着可能な言語のクラスについて
より詳しく調べることである.
定理$\ref{KTimesIsCH}$では便宜上$\classCH$困難と述べるに留まっているが,
\ref{subsection: counting hierarchy}節において定義した
$\quantC_{\log} B_{be}$にカープ帰着可能な言語のクラスについて困難であるといえる.
多項式階層が交替性機械によって特徴付けられるように,
計数階層も確率的な交替性機械を定義することによって特徴付けが可能である.
確率的交替性機械の交替回数を対数オーダーであるような機械によって認識される言語と等しい.
そのようなクラスが$\classCH$と$\classPSPACE$の間のどこに位置するかは以前不明である.

$g$について任意回連続微分可能と解析的の間の制限において,
常微分方程式の解の計算量がどうなるかは以前不明である.
特に導入でも述べたとおり$(\infty, \infty)$微分可能という制限のもとでどうなるか.

また前節でも述べたとおり, 構成的な形での定理 \ref{KTimesIsCH} については別稿にて扱う.
