\section{今後の課題}
一つ目の課題は$(\infty, k)$回連続微分可能の場合について,
$\classCH$以上の困難性を言えるのかという点である.
定理 $\ref{KTimesIsCH}$ では便宜上$\classCH$困難と述べるにとどまっているが,
\ref{subsection: counting hierarchy} 節において定義される
$\quantC_{\log} B_be$にカープ還元可能な言語のクラスについて困難であるといえる.
そのような言語クラスは, 交替性機械を確率的に動作するように拡張し,
交替回数が対数オーダーであるような機械によって認識される言語と等しい.
そのようなクラスについての先行研究は存在しておらず, 大変興味深い.
また導入でも述べたとおり無限微分可能という制限のもとでの,
常微分方程式の解の計算量については以前不明である.

演算子の計算量については進行中.
