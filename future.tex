\section{今後の課題}
1つ目の課題は$(\infty, k)$回連続微分可能という制限の元でのタイトな計算量を見つけることである.
定理$\ref{KTimesIsCH}$では便宜上$\classCH$困難と述べるに留まっているが,
少なくとも\ref{subsection: counting hierarchy}節において定義した
$\quantC_{\log} B_{be}$にカープ帰着可能な言語のクラスについて困難であることは示せる.
%また交替性機械を計数階層を特徴付けるに拡張したとき, 交替回数が入力の対数オーダーで抑えられる機械によって受理される言語クラスと一致する.
そのようなクラスが$\classCH$と$\classPSPACE$の間のどこに位置するかは以前不明である.

2つ目は$g$について任意回連続微分可能と解析的の間の制限の変化と
常微分方程式の解の計算量の関係は未知である.
導入でも述べたとおり$(\infty, \infty)$微分可能という制限のもとでどうなるか.

また前節でも述べたとおり, 構成的な形での定理 \ref{KTimesIsCH} については別稿にて扱う.
