\section{演算子の計算量}

定理\ref{DifferentiableIsPspace}, 
\ref{KTimesIsCH}はいづれも
関数$g$を多項式時間計算可能と仮定した上で
解$h$の計算量について述べている. 
しかし微分方程式を「解く」困難さ, 
すなわち与えられた$g$から$h$を求める演算子の計算量は
如何程であろうか. 
この問に答えるには先ず
実関数を実関数へ写す演算子の計算量を
定義することを要する. 

\newcommand{\OpIVP}{\mathit{ODE}}
\newcommand{\deltabox}{\delta _\square}
\newcommand{\deltaboxLip}{\delta _{\square \mathrm L}}
\newcommand{\classtwofont}[1]{\text{\bfseries \sffamily \upshape #1}}
\newcommand{\classFPSPACEtwo}{\classtwofont{FPSPACE}}
\newcommand{\classCHtwo}{\classtwofont{CH}}
\newcommand{\redMF}{\leq _{\mathrm{mf}}}
\newcommand{\redW}{\leq _{\mathrm W}}
\newcommand{\redSW}{\leq _{\mathrm{sW}}}
\newcommand{\classLip}{\mathrm C _{\mathrm L}}
\newcommand{\classC}{\mathrm C}

実数を入出力する関数の計算量を論ずるには, 
実数を文字列関数で表した. 
即ち$\R$の各元の名として文字列関数を使ったのであり, 
その対応を$\R$の表現という. 
同じように実関数を入出力する演算子の計算量を論ずるには, 
実関数を文字列関数で表す. 
すなわち, 
連続な実関数$h \colon [0,1] \to \R$の全体
$\classC [0, 1]$や, 
リプシッツ連続な実関数$g \colon [0, 1] \times [-1, 1] \to \R$の全体
$\classLip [[0, 1] \times [-1, 1]]$について, 
表現を指定すればよい. 
演算子の計算可能性や計算量は
その表現に依ることになるが, 
ここでは
\cite{kawamura2010operators}に従い, 
$\classC [0, 1]$の表現として$\deltabox$を, 
$\classLip [[0, 1] \times [-1, 1]]$の表現として$\deltaboxLip$を
それぞれ用いる. 
これらの表現では使われる文字列関数の長さが有界でないため, 
神託機械の時間・空間を測る方法を二階多項式を使って拡張する必要があるが, 
詳細は\cite{kawamura2010operators}を参照せられたい. 
$\deltabox$は関数空間$\classC [0, 1]$の表現として
或る意味で自然なものであることが判っており
\cite{}, 
また$\deltaboxLip$はそれと同様の表現にリプシッツ定数に関する情報を附加したものである. 

実関数$g \in \classLip [[0, 1] \times [-1, 1]]$を, 
\eqref{eq:ode}を満す$h \in \classC [0, 1]$に対応させる演算子$\OpIVP$を考える. 
$\OpIVP$は$\classLip [[0, 1] \times [-1, 1]]$から
$\classC [0, 1]$への部分写像である. 

\cite{kawamura2010operators}では表\ref{table:related}第三行の証明を
構成的に書き直すことで, 
$\OpIVP$が$(\deltaboxLip, \deltabox)$-$\classFPSPACEtwo$-$\redW$完全であることが
示されている. 
本稿の関心はこれを滑らかな入力に制限しても困難性が成立つかという問である. 
$\OpIVP$を$(\infty, k)$回連続微分可能な入力に制限したものを
$\OpIVP _k$と書くことにする ($k \in \N \cup \{\infty\}$). 

帰着の定義に$\redMF$, $\redW$, $\redSW$が
ほげほげ……

\begin{theorem}
\label{theorem: C1 constructive}
$\OpIVP _1$は$(\deltaboxLip, \deltabox)$-$\classFPSPACEtwo$-$\redW$完全. 
\end{theorem}

\begin{theorem}
\label{theorem: Ck constructive}
$\OpIVP _k$は$(\deltaboxLip, \deltabox)$-$\classFPSPACEtwo$に属し, 
$(\deltaboxLip, \deltabox)$-$\classCHtwo$-$\redW$困難. 
\end{theorem}

これらを示すには, 
定理\ref{DifferentiableIsPspace}, \ref{KTimesIsCH}の証明において
関数の構成に使われた情報が入力から容易に得られることを確かめればよく, 
新たな技巧を要しないから詳細は省略する. 
\cite{kawamura2010operators}で示されているように, 
定理\ref{DifferentiableIsPspace}, \ref{KTimesIsCH}はそれぞれ
定理\ref{theorem: C1 constructive}, \ref{theorem: Ck constructive}の
系として得られる. 

\begin{theorem}
$\OpIVP _\infty$は$(\deltaboxLip, \deltabox)$-$\classFPSPACEtwo$-$\redSW$完全. 
\end{theorem}

