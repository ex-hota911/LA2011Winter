\section{Complexity of Operations}

Both Theorems \ref{DifferentiableIsPspace} and \ref{KTimesIsCH}
state the complexity of the solution $h$ under the assumption 
that $g$ is polynomial-time computable.
But how hard is it to ``solve'' differential equations,
i.e., how complex is the operator that takes $g$ to $h$? 
To make this question precise,
we need to define the complexity of operators from real functions to real functions.

To discuss complexity of real functions,
we encoded real numbers by functions from strings to strings.
We used string functions as names of elements in $\R$. 
Such an encoding is called a \emph{representation} of $\R$.
In the same way, we represent real functions as string functions
to discuss complexity of real operators.
So we need to define representations as string functions of
class $\classC_{[0, 1]}$ of continuous real functions $h \colon [0,1] \to \R$ 
and class $\classLip_{[0, 1] \times [-1, 1]}$ of Lipschitz continuous real functions.
The choice of representations is important since the computability and complexity is depending on these representations.
Following \cite{kawamura2010operators},
we use $\deltabox$ as the representation of $\classC_{[0,1]}$ and $\deltaboxLip$ as the representation of $\classLip_{[0, 1] \times [-1, 1]}$.
It is known that is some sense $\deltabox$ an unique natural representation of $\classC_{[0, 1]}$ \cite{kawamura11:_funct_space_repres_and_polyn_time_comput}.
And $\deltaboxLip$ is a representation
added information about the Lipschitz constant to $\deltabox$.

Since size of string functions are not finite,
by restricting the amount of resources (time and space) of oracle Turing machine using second order polynomial,
we define second-order complexity classes
(e.g. $\classFPSPACEtwo$, polynomial-space computable),
reductions (e.g. $\redW$, polynomial-time Weihrauch reduction),
and hardness.
Combining it with the representations of real functions described above,
we can write theorems of this paper in constructive form.

Let $\OpIVP$ be an operator mapping a real function $g$ of class $\classLip_{[0, 1] \times [-1, 1]}$ to
the solution $h$ of \eqref{eq:ode} of class $\classC_{[0, 1]}$.
The operator $\OpIVP$ is a partial projection from $\classLip [[0, 1] \times [-1, 1]]$ to $\classC [0, 1]$.
In \cite[Theorem 4.9]{kawamura2010operators},
$(\deltaboxLip, \deltabox)$-$\classFPSPACEtwo$-$\redW$-completeness of $\OpIVP$ is proven
by rewriting the proof of the result of the third row in Table~\ref{table:related} in the constructive form.
In a similar way, Theorem~\ref{DifferentiableIsPspace} can be rewritten in the constructive form.
That is, let $\OpIVP _k$ as the operator $\OpIVP$ whose input is restricted to class $\classC^{(\infty, k)}$,

\begin{theorem}
\label{theorem: C1 constructive}
The operator $\OpIVP _1$ is $(\deltaboxLip, \deltabox)$-$\classFPSPACEtwo$-$\redW$-complete.
\end{theorem}

To show this theorem,
we need to verify that the information used to construct functions in the proof of Theorem~\ref{DifferentiableIsPspace}
can be acquired easily from inputs.
We omit the proof since it does not need any new technique.
This constructive theorem is stronger than non-constructive one \cite[Lemma 3.7 and 3.8]{kawamura2010operators}, 
i.e., Theorem~\ref{DifferentiableIsPspace} follows immediately
form Theorem~\ref{theorem: C1 constructive}.

The constructive version of Theorem~\ref{KTimesIsCH} is also true.
In fact for each $k \in \N$,
$\OpIVP _k$ is $(\deltaboxLip, \deltabox)$-$\classCHtwo$-$\redW$-hard.
We will treat it in another paper
since we need more discussion on how to handle hierarchy
to define $\classCHtwo$ in the flame work of \cite{kawamura2010operators}.

