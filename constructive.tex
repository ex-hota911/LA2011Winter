\section{Complexity of Operators}

Both Theorems \ref{DifferentiableIsPspace} and \ref{KTimesIsCH}
state the complexity of the solution $h$ under the assumption 
that $g$ is polynomial-time computable.
But how hard is it to ``solve'' differential equations,
i.e., how complex is the operator that takes $g$ to $h$? 
To make this question precise,
we need to define the complexity of operators from real functions to real functions.

Recall that, to discuss complexity of real functions,
we used string functions as names of elements in $\R$. 
Such an encoding is called a \emph{representation} of $\R$.
In the same way, 
we now want to encode real functions as string functions
to discuss complexity of real operators. 
In other words, we need to define representations of
the class $\classC _{[0, 1]}$ of continuous functions $h \colon [0,1] \to \R$ 
and class $\classLip _{[0, 1] \times [-1, 1]}$ of Lipschitz continuous functions $g \colon [0, 1] \times [-1, 1] \to \R$. 
The notions of computability and complexity depend on these representations.
Following \cite{kawamura2010operators},
we use $\deltabox$ as the representation of $\classC_{[0,1]}$ and $\deltaboxLip$ as the representation of $\classLip_{[0, 1] \times [-1, 1]}$.
It is known that 
$\deltabox$ is a unique canonical representation of $\classC_{[0, 1]}$ 
in a certain sense \cite{kawamura11:_funct_space_repres_and_polyn_time_comput}, 
and $\deltaboxLip$ is the representation defined by adding to $\deltabox$
the information on the Lipschitz constant.

Since these representations use string functions 
whose values have variable lengths,
we use \emph{second order polynomials}
to bound the amount of resources (time and space) of machines
\cite{kawamura2010operators}, 
and this leads to the definitions of second-order complexity classes
(e.g. $\classFPSPACEtwo$, polynomial-space computable),
reductions (e.g. $\redW$, polynomial-time Weihrauch reduction), 
and hardness.
Combining them with the representations of real functions described above,
we can restate the theorems in this paper in the constructive form as follows.

Let $\OpIVP$ be the operator mapping a real function $g \in \classLip_{[0, 1] \times [-1, 1]}$ to
the solution $h \in \classC_{[0, 1]}$ of \eqref{eq:ode}.
The operator $\OpIVP$ is a partial function from $\classLip _{[0, 1] \times [-1, 1]}$ to $\classC _{[0, 1]}$.
In \cite[Theorem 4.9]{kawamura2010operators}, the
$(\deltaboxLip, \deltabox)$-$\classFPSPACEtwo$-$\redW$-completeness of $\OpIVP$ is proven
by rewriting 
the proof of the results in the third row of Table~\ref{table:related} in the constructive form.
In a similar way, Theorem~\ref{DifferentiableIsPspace} can be rewritten in the constructive form.
That is, let $\OpIVP _k$ as the operator $\OpIVP$ whose input is restricted to class $\classC^{(\infty, k)}$. Then we have: 

\begin{theorem}
\label{theorem: C1 constructive}
The operator $\OpIVP _1$ is $(\deltaboxLip, \deltabox)$-$\classFPSPACEtwo$-$\redW$-complete.
\end{theorem}

To show this theorem,
we need to verify that the information used to construct functions in the proof of Theorem~\ref{DifferentiableIsPspace}
can be acquired easily from inputs.
We omit the proof since it does not need any new technique.
This constructive form implies the non-constructive form \cite[Lemma 3.7 and 3.8]{kawamura2010operators}; 
thus, Theorem~\ref{DifferentiableIsPspace} is a corollary of 
Theorem~\ref{theorem: C1 constructive}.

The constructive version of Theorem~\ref{KTimesIsCH} is also true: 
for each $k \in \N$,
the restricted operator $\OpIVP _k$ is 
$(\deltaboxLip, \deltabox)$-$\classCHtwo$-$\redW$-hard.
But the formulation of this 
second-order version $\classCHtwo$ of the counting hierarchy 
requires some discussion on relativized computation, 
which will appear in a forthcoming paper. 
