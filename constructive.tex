\section{演算子の計算量}

Both Theorem~\ref{DifferentiableIsPspace} and \ref{KTimesIsCH}
states the complexity of the solution $h$ under the assumption 
that $g$ is polynomial-time computable.
How complex can it be to ``solve'' differential equations,
i.e., how complex is the operator computing $h$ by given $g$?
To make an answer to this problem,
we need to define complexity of operators from real functions to real functions.

To discuss complexity of real functions,
we encode real numbers by functions from string to string.
We use string functions as names of elements in $\R$,
such relation is called representation of $\R$.
In the same way, we represent real functions as string functions
to discuss complexity of real operators.
So we need to define representations as string functions of
class $\classC_{[0, 1]}$ of continuous real functions $h \colon [0,1] \to \R$ 
and class $\classLip_{[0, 1] \times [-1, 1]}$ of Lipschitz continuous real functions.
The choice of representations is important since the computability and complexity is depending on these representations.
Following \cite{kawamura2010operators},
we use $\deltabox$ as the representation of $\classC_{[0,1]}$ and $\deltaboxLip$ as the representation of $\classLip_{[0, 1] \times [-1, 1]}$.
It is known that is some sense $\deltabox$ an unique natural representation of $\classC_{[0, 1]}$ \cite{kawamura11:_funct_space_repres_and_polyn_time_comput}.
And $\deltaboxLip$ is a representation
added information about the Lipschitz constant to $\deltabox$.

Since size of string functions are not finite,
by restricting the amount of resources (time and space) of oracle Turing machine using second order polynomial,
we define second-order complexity classes
(e.g. $\classFPSPACEtwo$, polynomial-space computable),
reductions (e.g. $\redW$, polynomial-time Weihrauch reduction),
and hardness.
Combining it with the representations of real functions described above,
we can write theorems of this paper in constructive form.

Let $\OpIVP$ be an operator mapping a real function $g$ of class $\classLip_{[0, 1] \times [-1, 1]}$ to
the solution $h$ of \eqref{eq:ode} of class $\classC_{[0, 1]}$.
The operator $\OpIVP$ is a partial projection from $\classLip [[0, 1] \times [-1, 1]]$ to $\classC [0, 1]$.
\cite[定理4.9]{kawamura2010operators}では表\ref{table:related}第三行の証明を
構成的に書き直すことで, 
$\OpIVP$が$(\deltaboxLip, \deltabox)$-$\classFPSPACEtwo$-$\redW$完全であることが
示された. 
本稿の定理\ref{DifferentiableIsPspace}も同じように
構成的に書き直すことができる. 
即ち$\OpIVP$を$(\infty, k)$回連続微分可能な入力に制限したものを
$\OpIVP _k$と書くと, 

\begin{theorem}
\label{theorem: C1 constructive}
$\OpIVP _1$は$(\deltaboxLip, \deltabox)$-$\classFPSPACEtwo$-$\redW$完全. 
\end{theorem}

これを示すには, 
定理\ref{DifferentiableIsPspace}の証明において
関数の構成に使われた情報が入力から容易に得られることを確かめればよく, 
新たな技巧を要しないから詳細は省略する. 
この構成的な主張は非構成的な主張よりも
強いものであり\cite[補題3.7, 3.8]{kawamura2010operators}, 
定理\ref{DifferentiableIsPspace}は
定理\ref{theorem: C1 constructive}の系として従う. 

% \begin{theorem}
% \label{theorem: Ck constructive}
% \end{theorem}

なお定理\ref{KTimesIsCH}も同様に構成的な形で成立ち, 
各$k \in \N$について
$\OpIVP _k$は$(\deltaboxLip, \deltabox)$-$\classCHtwo$-$\redW$困難であるが, 
この\cite{kawamura2010operators}の枠組における$\classCHtwo$を定義するには
相対化の扱いについて今少しの議論を要するので別稿で扱う. 

% ところで, より強い帰着である
% 強ワイラオホ帰着を用いると

% \begin{theorem}
% $\OpIVP _\infty$は$(\deltaboxLip, \deltabox)$-$\classFPSPACEtwo$-$\redSW$完全. 
% \end{theorem}
