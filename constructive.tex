\section{演算子の計算量}

定理\ref{DifferentiableIsPspace}, 
\ref{KTimesIsCH}はいづれも
関数$g$を多項式時間計算可能と仮定した上で
解$h$の計算量について述べている. 
しかし微分方程式を「解く」困難さ, 
すなわち与えられた$g$から$h$を求める演算子の計算量は
如何であろうか. 
この問に答えるにはまず
実関数を実関数へ写す演算子の計算量を
定義することを要する. 

実数を入出力する関数の計算量を論ずるには, 
実数を文字列関数で表した. 
即ち$\R$の各元の名として文字列関数を使ったのであり, 
その対応を$\R$の表現という. 
同じように実関数を入出力する演算子の計算量を論ずるには, 
実関数を文字列関数で表す. 
つまり
連続な実関数$h \colon [0,1] \to \R$の空間$\classC [0, 1]$や, 
Lipschitz連続な実関数$g \colon [0, 1] \times [-1, 1] \to \R$の空間
$\classLip [[0, 1] \times [-1, 1]]$について, 
表現を指定すればよい. 
演算子の計算可能性や計算量は
その表現に依ることになるが, 
ここでは
\cite{kawamura2010operators}に従い, 
$\classC [0, 1]$の表現として$\deltabox$を, 
$\classLip [[0, 1] \times [-1, 1]]$の表現として$\deltaboxLip$を
それぞれ用いる. 
$\deltabox$は関数空間$\classC [0, 1]$の表現として
或る意味で自然な唯一のものであることが判っており
\cite{kawamura11:_funct_space_repres_and_polyn_time_comput}, 
また$\deltaboxLip$は$\deltabox$にLipschitz定数の情報を附加した表現である. 

これらの表現では使われる文字列関数の長さが有界でないため, 
神託機械の時間・空間を測る方法を二階多項式を使って拡張し, 
これに基いて多項式空間$\classFPSPACEtwo$などの計算量クラスや, 
多項式時間Weihrauch帰着$\redW$などの帰着, 
その下での困難性を定義する\cite{kawamura2010operators}. 
この枠組を上述の実関数の表現と組合せることで, 
本稿の結果も以下の如く構成的な形で述べることができる. 

実関数$g \in \classLip [[0, 1] \times [-1, 1]]$を, 
\eqref{eq:ode}の解$h \in \classC [0, 1]$に対応させる演算子$\OpIVP$を考える. 
$\OpIVP$は$\classLip [[0, 1] \times [-1, 1]]$から
$\classC [0, 1]$への部分写像である. 
\cite[定理4.9]{kawamura2010operators}では表\ref{table:related}第三行の証明を
構成的に書き直すことで, 
$\OpIVP$が$(\deltaboxLip, \deltabox)$-$\classFPSPACEtwo$-$\redW$完全であることが
示された. 
本稿の定理\ref{DifferentiableIsPspace}も同じように
構成的に書き直すことができる. 
即ち$\OpIVP$を$(\infty, k)$回連続微分可能な入力に制限したものを
$\OpIVP _k$と書くと, 

\begin{theorem}
\label{theorem: C1 constructive}
$\OpIVP _1$は$(\deltaboxLip, \deltabox)$-$\classFPSPACEtwo$-$\redW$完全. 
\end{theorem}

これを示すには, 
定理\ref{DifferentiableIsPspace}の証明において
関数の構成に使われた情報が入力から容易に得られることを確かめればよく, 
新たな技巧を要しないから詳細は省略する. 
この構成的な主張は非構成的な主張よりも
強いものであり\cite[補題3.7, 3.8]{kawamura2010operators}, 
定理\ref{DifferentiableIsPspace}は
定理\ref{theorem: C1 constructive}の系として従う. 

% \begin{theorem}
% \label{theorem: Ck constructive}
% \end{theorem}

なお定理\ref{KTimesIsCH}も同様に構成的な形で成立ち, 
各$k \in \N$について
$\OpIVP _k$は$(\deltaboxLip, \deltabox)$-$\classCHtwo$-$\redW$困難であるが, 
この\cite{kawamura2010operators}の枠組における$\classCHtwo$を定義するには
相対化の扱いについて今少しの議論を要するので別稿で扱う. 

% ところで, より強い帰着である
% 強ワイラオホ帰着を用いると

% \begin{theorem}
% $\OpIVP _\infty$は$(\deltaboxLip, \deltabox)$-$\classFPSPACEtwo$-$\redSW$完全. 
% \end{theorem}
