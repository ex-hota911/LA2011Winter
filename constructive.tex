\section{演算子の計算量}

Both Theorem~\ref{DifferentiableIsPspace} and \ref{KTimesIsCH}
states the complexity of the solution $h$ under the assumption 
that $g$ is polynomial-time computable.
How complex can it be to ``solve'' differential equations,
i.e., how complex is the operator computing $h$ by given $g$?
To make an answer to this problem,
we need to define complexity of operators from real functions to real functions.

To discuss complexity of real functions,
we encode real numbers by functions from string to string.
We use string functions as names of elements in $\R$,
such relation is called representation of $\R$.
In the same way, we represent real functions as string functions
to discuss complexity of real operators.
つまり
連続な実関数$h \colon [0,1] \to \R$の空間$\classC [0, 1]$や, 
Lipschitz連続な実関数$g \colon [0, 1] \times [-1, 1] \to \R$の空間
$\classLip [[0, 1] \times [-1, 1]]$について, 
表現を指定すればよい. 
The computability and complexity is depending on the representation,
so we adopt \cite{kawamura2010operators}
using $\deltabox$ as representation of continuous functions $f \colon [0,1] \to \R$ and $\deltaboxLip$ as representation of Lipschitz continuous 
functions $g \colon [0, 1] \times [-1, 1] \to \R$.
It is known that $\deltabox$ is some unique natural representation of $\classC_{[0, 1]}$ \cite{kawamura11:_funct_space_repres_and_polyn_time_comput}, 
and $\deltaboxLip$ is a representation
added the information about Lipschitz constants to $\deltabox$.

Since the length of string functions are not finite,
we generate the way to calculate time and space resource of oracle Turing machine
using second order polynomial.

これに基いて多項式空間$\classFPSPACEtwo$などの計算量クラスや, 
多項式時間Weihrauch帰着$\redW$などの帰着, 
その下での困難性を定義する\cite{kawamura2010operators}. 
この枠組を上述の実関数の表現と組合せることで, 
本稿の結果も以下の如く構成的な形で述べることができる. 

実関数$g \in \classLip [[0, 1] \times [-1, 1]]$を, 
\eqref{eq:ode}の解$h \in \classC [0, 1]$に対応させる演算子$\OpIVP$を考える. 
$\OpIVP$は$\classLip [[0, 1] \times [-1, 1]]$から
$\classC [0, 1]$への部分写像である. 
\cite[定理4.9]{kawamura2010operators}では表\ref{table:related}第三行の証明を
構成的に書き直すことで, 
$\OpIVP$が$(\deltaboxLip, \deltabox)$-$\classFPSPACEtwo$-$\redW$完全であることが
示された. 
本稿の定理\ref{DifferentiableIsPspace}も同じように
構成的に書き直すことができる. 
即ち$\OpIVP$を$(\infty, k)$回連続微分可能な入力に制限したものを
$\OpIVP _k$と書くと, 

\begin{theorem}
\label{theorem: C1 constructive}
$\OpIVP _1$は$(\deltaboxLip, \deltabox)$-$\classFPSPACEtwo$-$\redW$完全. 
\end{theorem}

これを示すには, 
定理\ref{DifferentiableIsPspace}の証明において
関数の構成に使われた情報が入力から容易に得られることを確かめればよく, 
新たな技巧を要しないから詳細は省略する. 
この構成的な主張は非構成的な主張よりも
強いものであり\cite[補題3.7, 3.8]{kawamura2010operators}, 
定理\ref{DifferentiableIsPspace}は
定理\ref{theorem: C1 constructive}の系として従う. 

% \begin{theorem}
% \label{theorem: Ck constructive}
% \end{theorem}

なお定理\ref{KTimesIsCH}も同様に構成的な形で成立ち, 
各$k \in \N$について
$\OpIVP _k$は$(\deltaboxLip, \deltabox)$-$\classCHtwo$-$\redW$困難であるが, 
この\cite{kawamura2010operators}の枠組における$\classCHtwo$を定義するには
相対化の扱いについて今少しの議論を要するので別稿で扱う. 

% ところで, より強い帰着である
% 強ワイラオホ帰着を用いると

% \begin{theorem}
% $\OpIVP _\infty$は$(\deltaboxLip, \deltabox)$-$\classFPSPACEtwo$-$\redSW$完全. 
% \end{theorem}
