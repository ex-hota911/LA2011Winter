\begin{abstract}
 常微分方程式 $h(0)=0,\ h'(t) = g(t, h(t))$ の解 $h$ の計算量と,
 関数 $g$ の計算量及び制限の関係は, 常微分方程式を数値的に解くことの
 本質的な難しさを表しているとして調べられている.
 河村は2010年の論文の中で, $g$ を Lipschitz 条件を満たす多項式時間計算可能な関数に限定した時でも
 解$h$が$\classPSPACE$困難になりうるという結果を示しているが,
 本稿ではさらに$g$を滑らかな関数への制限による$h$の計算量の下限の変化を観察した.
 結果, 1回連続微分可能に限ったとしても,
 解は$\classPSPACE$困難になりうるが,
 2回以上微分可能な$g$においては,
 解$h$は計数階層 $\classCH$ について困難であることを示した.
\end{abstract}

