\begin{abstract}
 常微分方程式 $h(0)=0,\ h'(t) = g(t, h(t))$ の解 $h$ の計算量と,
 関数 $g$ の計算量及び制限の関係は, 常微分方程式を数値的に解くことの
 本質的な難しさを表しているとして調べられている.
 本稿では河村が2010年の論文の中で Lipschitz 条件を満たす多項式時間計算可能な関数
 $g$ の常微分方程式の解が $\PSPACE$ 困難たりうるという結果をしめすために用いた手法を,
 微分可能な $g$ に拡張する.
 これにより多項式時間計算可能で1回連続微分可能な関数の常微分方程式が,
 $\PSPACE$ 困難な解を持ちうることがわかる. 
 また任意の $k$ について, 多項式時間計算可能で $k$ 回微分可能な関数の常微分方程式は,
 本稿で定義される計算量 $\DIVPlog$ について困難な解を持ちうることをしめす.
 $\DIVPlog = \PSPACE$ であるかどうかは未解決である.
\end{abstract}

