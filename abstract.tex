\begin{abstract}
 常微分方程式 $h(0)=0,\ h'(t) = g(t, h(t))$ の解 $h$ の計算量と,
 関数 $g$ の計算量及び制限の関係は, 常微分方程式を数値解析的に解くことの
 本質的な難しさを表しているとして調べられている.
 本稿では河村が2010年の論文の中で Lipschitz 条件を満たす多項式時間計算可能な関数
 $g$ の常微分方程式の解が\PSPACE 困難たりうる結果をしめすために用いた手法を,
 微分可能な $g$ について拡張する.
 多項式時間計算可能で1回連続的微分可能な関数の常微分方程式は,
 \PSPACE 困難な解を持ちうること,
 多項式時間計算可能で任意回微分可能関数の常微分方程式は,
 本稿の中で定義される計算量 \DIVPlog 困難な解を持ちうることをしめす.
 \DIVPlog $=$ \PSPACE であるかどうかはは未解決である.
\end{abstract}

